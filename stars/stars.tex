\documentclass[a4paper,11pt]{article}
\usepackage{graphicx}
\usepackage{caption}
\usepackage{enumitem}
\usepackage{multicol}
\usepackage{mathtools}
\usepackage{amsmath,amsthm,amssymb,cancel,bm}
\usepackage{floatrow}
\setcounter{tocdepth}{2}
\usepackage{geometry}
\geometry{total={210mm,297mm},
left=25mm,right=25mm,%
bindingoffset=0mm, top=20mm,bottom=20mm}
\newcommand{\linia}{\rule{\linewidth}{0.5pt}}

\usepackage{hyperref}
\hypersetup{colorlinks=true,linkcolor=blue,citecolor=green,filecolor=cyan,urlcolor=magenta}

% my own titles
\makeatletter
\renewcommand{\maketitle}{
\begin{center}
\vspace{2ex}
{\huge \textsc{\@title}}
\vspace{1ex}
\\
\linia\\
\@author
\vspace{4ex}
\end{center}
}
\makeatother

% custom footers and headers
\usepackage{fancyhdr,lastpage}
\pagestyle{fancy}
\lhead{}
\chead{}
\rhead{}
\renewcommand{\headrulewidth}{0pt}
\lfoot{General Qualifying Exam Solutions}
\cfoot{}
\rfoot{Page \thepage\ /\ \pageref*{LastPage}}

% --------------------------------------------------------------
%
%                           TITLE PAGE
%
% --------------------------------------------------------------

\begin{document}
\hfill{\textit{Last modified \today}}
\title{General Qualifying Exam Solutions: Stellar Astrophysics}
\author{Jessica Campbell, Dunlap Institute for Astronomy \& Astrophysics (UofT)}
\date{\today}
\maketitle

\tableofcontents



% --------------------------------------------------------------
%
%
%                              STARS 
%
%
% --------------------------------------------------------------

\newpage
\section{Stellar Astrophysics}

% --------------------------------------------------------------
%               1. 
% --------------------------------------------------------------

\subsection{Question 1}

Sketch out a Hertsprung-Russell diagram. Indicate where on the main sequence different spectral classes lie. Draw and describe the post main-sequence tracks of both low- and high-mass stars.

\subsubsection{Short answer}

Answer.

\subsubsection{Additional context}

Additional context.

\begin{itemize}
    \item How does metallicity affect this diagram?
    \item Elaborate on the life sequence of low-mass stars.
    \item Elaborate on the life sequence of high-mass stars.
    \item Which direction does radius increase?
    \item What's the functional form for how radius increases as a function of temperature and luminosity?
    \item Why can you use the tip of the red giant branch for distance determination?
    \item How can you use the H-R diagram to determine ages? Which of the two techniques is more accurate?
\end{itemize}

% --------------------------------------------------------------
%               2. 
% --------------------------------------------------------------

\newpage
\subsection{Question 2}

Sketch a plot of radius versus mass for various ``cold'' objects made of normal matter, including planets, brown dwarfs and white dwarfs. Explain the mass-size relationship for rocky and
gaseous objects. Why is there an upper mass limit?

\subsubsection{Short answer}

Answer.

\subsubsection{Additional context}

Additional context.

\subsubsection{Follow-up Questions}

\begin{itemize}
    \item How do you calculate the Chandrasekhar mass limit?
    \item Why is Saturn smaller than Jupiter? Or, why do we see a range of radii in extrasolar planets (e.g., hot Jupiters)?
\end{itemize}

% --------------------------------------------------------------
%               3. 
% --------------------------------------------------------------

\newpage
\subsection{Question 3}

Describe the physical conditions that lead to the formation of absorption lines in stars' spectra. What leads to emission lines?

\subsubsection{Short answer}

Answer.

\subsubsection{Additional context}

Additional context.

\subsubsection{Follow-up Questions}

\begin{itemize}
    \item Why aren't emission and absorption lines delta functions?
    \item How does this relate to population levels and excitation temperatures?
    \item Are there emission lines in the Sun? Why is there emission from the Calcium doublet?
    \item Write down the heat transfer equation. What do solutions look like?
\end{itemize}

% --------------------------------------------------------------
%               4. 
% --------------------------------------------------------------

\newpage
\subsection{Question 4}

Describe these important sources of stellar opacity: electron scattering, free-free, bound-free, and the H$^-$ ion.

\subsubsection{Short answer}

Answer.

\subsubsection{Additional context}

Additional context.

% --------------------------------------------------------------
%               5. 
% --------------------------------------------------------------

\newpage
\subsection{Question 5}

Describe the processes that can cause pulsations in a star’s luminosity, and provide at least one example of a class of stellar pulsation.

\subsubsection{Short answer}

Answer.

\subsubsection{Additional context}

Additional context.

\subsubsection{Follow-up Questions}

\begin{itemize}
    \item What about the instability strip? RR Lyrae?
    \item What is the period-luminosity relation?
    \item What is the form of the period-luminosity relation?
    \item How would you derive the time scale of pressure waves in a star?
    \item How would you order-of-magnitude estimate the period for a pulsation?
\end{itemize}

% --------------------------------------------------------------
%               6. 
% --------------------------------------------------------------

\newpage
\subsection{Question 6}

Briefly describe the sources of thermal energy for stars and planets.

\subsubsection{Short answer}

Answer.

\subsubsection{Additional context}

Additional context.

\subsubsection{Follow-up Questions}

\begin{itemize}
    \item Why do different nuclear reaction pathways have different temperature sensitivities?
    \item If I assume a constant core temperature on the main sequence, how does stellar radius depend on mass?
    \item What are some other thermal sources, like say for neutron stars?
\end{itemize}

% --------------------------------------------------------------
%               7. 
% --------------------------------------------------------------

\newpage
\subsection{Question 7}

Describe the process by which supernovae produce light. Why are Type Ia supernovae generally brighter than Type II events?

\subsubsection{Short answer}

Answer.

\subsubsection{Additional context}

Additional context.

\subsubsection{Follow-up Questions}

\begin{itemize}
    \item When a star goes supernova, how much of the luminous energy generated at the rebound is available for heating the gas? As in, where does the heat come from?
    \item Are there cases where the rebound shock wave can't blow up the star? Why, and what happens then?
\end{itemize}

% --------------------------------------------------------------
%               8. 
% --------------------------------------------------------------

\newpage
\subsection{Question 8}

Describe the condition for a star’s envelope to become convective. Why are low mass stars convective in their outer envelopes while high mass stars are convective in their inner cores?

\subsubsection{Short answer}

Answer.

\subsubsection{Additional context}

Additional context.

\subsubsection{Follow-up Questions}

\begin{itemize}
    \item How do we know that the Sun's outer envelope is convective?
    \item How far into the surface of the Sun does the convective zone permeate? How can we measure this?
\end{itemize}

% --------------------------------------------------------------
%               9. 
% --------------------------------------------------------------

\newpage
\subsection{Question 9}

What is Eddington’s luminosity limit? Explain why this limit is important for the properties and lifetimes of massive stars.

\subsubsection{Short answer}

Answer.

\subsubsection{Additional context}

Additional context.

\subsubsection{Follow-up Questions}

\begin{itemize}
    \item Draw a force diagram of what’s happening.
    \item What particles experience gravity the most?
    \item What particles experience photon pressure the most?
\end{itemize}

% --------------------------------------------------------------
%               10. 
% --------------------------------------------------------------

\newpage
\subsection{Question 10}

Explain why we know what the Sun’s central temperature ought to be, and how we know what it actually is.

\subsubsection{Short answer}

Answer.

\subsubsection{Additional context}

Additional context.

% --------------------------------------------------------------
%               11. 
% --------------------------------------------------------------

\newpage
\subsection{Question 11}

Which have higher central pressure, high-mass or low-mass main-sequence stars? Roughly, what is their mass-radius relation? Derive this.

\subsubsection{Short answer}

Answer.

\subsubsection{Additional context}

Additional context.
\subsubsection{Follow-up Questions}

\begin{itemize}
    \item How would we actually know the central pressure?
    \item What properties can we measure to test models of stellar structure?
\end{itemize}

% --------------------------------------------------------------
%               12. 
% --------------------------------------------------------------

\newpage
\subsection{Question 12}

Sketch the SED of an O, A, G, M, and T star. Give defining spectral characteristics, such as the Balmer lines and Balmer jump and Calcium doublets, and describe physically.

\subsubsection{Short answer}

Answer.

\subsubsection{Additional context}

Additional context.

\subsubsection{Follow-up Questions}

\begin{itemize}
    \item Are there emission lines?
    \item What molecular lines are in the Sun?
    \item What important lines are there are much longer wavelengths than those in the optical?
    \item What if the A star had a protoplanetary disk?
    \item What is the significance of a $\lambda$F$_\lambda$ (or $\nu$F$_\nu$) spectrum?
    \item How can the relative height of the stellar vs disk bumps change? (i.e., total energy of the system cannot change; dust bump can't be higher than star bump without extinction.)
\end{itemize}

% --------------------------------------------------------------
%               13. 
% --------------------------------------------------------------

\newpage
\subsection{Question 13}

What can be learned about young stars (T Tauri and pre-main-sequence stars) from an analysis of their spectral features?

\subsubsection{Short answer}

Answer.

\subsubsection{Additional context}

Additional context.

\subsubsection{Follow-up Questions}

\begin{itemize}
    \item How does the spectrum change as planets start to form?
\end{itemize}

% --------------------------------------------------------------
%               14. 
% --------------------------------------------------------------

\newpage
\subsection{Question 14}

Sketch the spectral energy distribution (SED) of a T Tauri star surrounded by a protoplanetary disk. How would the SED change: (a) if the disk develops a large inner hole, (b) if the dust grains in the disk grow in size by agglomeration (with the same total mass)?

\subsubsection{Short answer}

Answer.

\subsubsection{Additional context}

Additional context.

% --------------------------------------------------------------
%               15. 
% --------------------------------------------------------------

\newpage
\subsection{Question 15}

What are the primary origins of the heat lost to space by infrared luminosity of Jupiter, Earth, and Io?

\subsubsection{Short answer}

Answer.

\subsubsection{Additional context}

Additional context.

% --------------------------------------------------------------
%               16. 
% --------------------------------------------------------------

\newpage
\subsection{Question 16}

Explain the observational problem of radius inflation for hot Jupiters and describe two possible solutions.

\subsubsection{Short answer}

Answer.

\subsubsection{Additional context}

Additional context.

% --------------------------------------------------------------
%               17. 
% --------------------------------------------------------------

\newpage
\subsection{Question 17}

Explain the effects of an atmosphere on a planet’s surface temperature and the position of the “habitable zone”. What special considerations must one make for habitability around M-type stars?

\subsubsection{Short answer}

Answer.

\subsubsection{Additional context}

Additional context.

% --------------------------------------------------------------
%               18. 
% --------------------------------------------------------------

\newpage
\subsection{Question 18}

Explain the process of nuclear fusion and give two examples of important fusion processes that affect the lives of stars.

\subsubsection{Short answer}

Answer.

\subsubsection{Additional context}

Additional context.

% --------------------------------------------------------------
%               19. 
% --------------------------------------------------------------

\newpage
\subsection{Question 19}

What is Fermi’s Paradox? Explain its logic and assess the current state of the Paradox in light of modern knowledge.

\subsubsection{Short answer}

Answer.

\subsubsection{Additional context}

Additional context.

\subsubsection{Follow-up Questions}

\begin{itemize}
    \item What percentage of stars have planets?
\end{itemize}

% --------------------------------------------------------------
%               20. 
% --------------------------------------------------------------

\newpage
\subsection{Question 20}

The so-called r- and s- processes are mechanisms that produce elements heavier than iron. Describe these mechanisms and evidence for them from abundance patterns. Where is the r-process thought to act?

\subsubsection{Short answer}

Answer.

\subsubsection{Additional context}

Additional context.

% --------------------------------------------------------------
%               Resources 
% --------------------------------------------------------------

\newpage
\subsection{Resources}

\begin{itemize}
    \item The Fundamentals of Stellar Astrophysics, Collins (2003)
    \item Stellar Structure and Evolution, Kippenhahn, Weigert \& Weiss (2012)
    \item Evolution of Stars and Stellar Populations, Salaris \& Cassisi (2005)
    \item The Astronomical Reach of Fundamental Physics, Burrows \& Ostriker (2014)
    \item Opacity, Huebner \& Barfield (2014)
    \item Radiative Processes in Astrophysics, Rybicky \& Lightman (1979)
    \item Understanding Variable Stars, Percy (2007)
\end{itemize}

\end{document}

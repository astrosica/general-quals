\documentclass[a4paper,11pt]{article}
\usepackage{graphicx}
\usepackage{caption}
\usepackage{enumitem}
\usepackage{multicol}
\usepackage{mathtools}
\usepackage{amsmath,amsthm,amssymb,cancel,bm}
\usepackage{floatrow}
\setcounter{tocdepth}{2}
\usepackage{geometry}
\geometry{total={210mm,297mm},
left=25mm,right=25mm,%
bindingoffset=0mm, top=20mm,bottom=20mm}
\newcommand{\linia}{\rule{\linewidth}{0.5pt}}

\usepackage{hyperref}
\hypersetup{colorlinks=true,linkcolor=blue,citecolor=green,filecolor=cyan,urlcolor=magenta}

% my own titles
\makeatletter
\renewcommand{\maketitle}{
\begin{center}
\vspace{2ex}
{\huge \textsc{\@title}}
\vspace{1ex}
\\
\linia\\
\@author
\vspace{4ex}
\end{center}
}
\makeatother

% custom footers and headers
\usepackage{fancyhdr,lastpage}
\pagestyle{fancy}
\lhead{}
\chead{}
\rhead{}
\renewcommand{\headrulewidth}{0pt}
\lfoot{General Qualifying Exam Solutions}
\cfoot{}
\rfoot{Page \thepage\ /\ \pageref*{LastPage}}

% --------------------------------------------------------------
%
%                           TITLE PAGE
%
% --------------------------------------------------------------

\begin{document}
\hfill{\textit{Last modified \today}}
\title{General Qualifying Exam Solutions: Appendix}
\author{Jessica Campbell, Dunlap Institute for Astronomy \& Astrophysics (UofT)}
\date{\today}
\maketitle

\tableofcontents

%%%%%%%%%%%%%%%%%%%%%%%%%%%%%%%%%%%%%%%%%%%%%%%%%%%%%%%%
%                                                      %
%                                                      %
%                      APPENDIX                        %
%                                                      %
%                                                      %
%%%%%%%%%%%%%%%%%%%%%%%%%%%%%%%%%%%%%%%%%%%%%%%%%%%%%%%%

\newpage
\section{Appendix}

%%%%%%%%%%%%%%%%%%%%%%%%%%%%%%%%%%%%%%%%%%%%%%%%%%%%%%%%
%                                                      %
%                                                      %
%                      ACRONYMS                        %
%                                                      %
%                                                      %
%%%%%%%%%%%%%%%%%%%%%%%%%%%%%%%%%%%%%%%%%%%%%%%%%%%%%%%%
\subsection{Acronyms}

{\noindent}\textbf{AGN}: active galactic nuclei

{\noindent}\textbf{AU}: astronomical unit

{\noindent}\textbf{BAOs:} baryonic acoustic oscillations

{\noindent}\textbf{BB:} Big Bang or blackbody

{\noindent}\textbf{BCD:} blue compact dwarf

{\noindent}\textbf{BBN:} Big Bang nucleosynthesis

{\noindent}\textbf{bf:} bound-free

{\noindent}\textbf{BH:} black hole

{\noindent}\textbf{CCD:} charged couple device

{\noindent}\textbf{CCD:} colour-colour diagram

{\noindent}\textbf{CIB:} Cosmic Infrared Background

{\noindent}\textbf{CMB:} Cosmic Microwave Background

{\noindent}\textbf{CMD:} colour-magnitude diagram

{\noindent}\textbf{CNB:} Cosmic Neutrino Background

{\noindent}\textbf{COBE:} COsmic Background Explorer

{\noindent}\textbf{CDM:} cold dark matter

{\noindent}\textbf{cE:} compact elliptical

{\noindent}\textbf{CIB:} cosmic infrared background

{\noindent}\textbf{CMB:} cosmic microwave background

{\noindent}\textbf{CR:} cosmic ray

{\noindent}\textbf{CRB:} cosmic radio background

{\noindent}\textbf{CGB:} cosmic gamma-ray background

{\noindent}\textbf{CUVOB:} cosmic ultraviolet/optical background

{\noindent}\textbf{CXB:} cosmic x-ray background

{\noindent}\textbf{DE:} dark energy

{\noindent}\textbf{dE:} dwarf elliptical

{\noindent}\textbf{DM:} dark matter

{\noindent}\textbf{DM:} distance modulus

{\noindent}\textbf{dSph:} dwarf spheroidal

{\noindent}\textbf{E:} elliptical

{\noindent}\textbf{EoR:} epoch of reionization

{\noindent}\textbf{ff:} free-free

{\noindent}\textbf{FIR:} far infrared

{\noindent}\textbf{GC:} galactic center

{\noindent}\textbf{GC:} globular cluster

{\noindent}\textbf{gE:} giant elliptical

{\noindent}\textbf{GP:} Gunn-Peterson

{\noindent}\textbf{GUT:} grand unified theory

{\noindent}\textbf{HRD}: Hertzsprung-Russel diagram

{\noindent}\textbf{HST}: Hubble Space Telescope

{\noindent}\textbf{ICM:} intracluster medium

{\noindent}\textbf{IGM:} intergalactic medium

{\noindent}\textbf{IMF:} initial mass function

{\noindent}\textbf{IR:} infrared

{\noindent}\textbf{Irr:} irregular

{\noindent}\textbf{ISM:} interstellar medium

{\noindent}\textbf{LAE:} Lyman-alpha emitter

{\noindent}\textbf{LBG:} Lyman break galaxy

{\noindent}\textbf{LG:} Local Group

{\noindent}\textbf{LMC:} Large Magellanic Cloud

{\noindent}\textbf{LTE:} local thermodynamic equilibrium

{\noindent}\textbf{MACHO:} massive compact halo objects

{\noindent}\textbf{MW:} Milky Way

{\noindent}\textbf{MWG:} Milky Way Galaxy

{\noindent}\textbf{NLTE:} non-local thermodynamic equilibrium

{\noindent}\textbf{NS:} neutron star

{\noindent}\textbf{pc}: parsec

{\noindent}\textbf{PGM}: pre-galactic medium

{\noindent}\textbf{PL}: period-luminosity

{\noindent}\textbf{PSF} point-spread function

{\noindent}\textbf{QFT:} quantum field theory

{\noindent}\textbf{QSO:} quasi-stellar object

{\noindent}\textbf{RJ:} Rayleigh-Jeans

{\noindent}\textbf{RT:} radiative transfer

{\noindent}\textbf{RTE:} radiative transfer equation

{\noindent}\textbf{S:} spiral

{\noindent}\textbf{SB:} barred spiral

{\noindent}\textbf{SDSS:} Sloan Digitial Sky Survey

{\noindent}\textbf{SED:} spectral energy distribution

{\noindent}\textbf{SF:} star formation

{\noindent}\textbf{SFR:} star formation rate

{\noindent}\textbf{SFRD:} star formation rate density

{\noindent}\textbf{SMC:} Small Magellanic Cloud

{\noindent}\textbf{SMBH:} supermassive black hole

{\noindent}\textbf{SMG:} sub-millimeter galaxy

{\noindent}\textbf{S/N:} signal-to-noise

{\noindent}\textbf{SNR:} signal-to-noise ratio

{\noindent}\textbf{SZ:} Sunyaev-Zeldovich

{\noindent}\textbf{SZE:} Sunyaev-Zeldovich effect

{\noindent}\textbf{TE:} thermal equilibrium

{\noindent}\textbf{TE:} thermodynamic equilibrium

{\noindent}\textbf{VLBI:} Very Long Baseline Interferometry

{\noindent}\textbf{WD:} white dwarf

{\noindent}\textbf{WIMP:} weakly interacting massive particle

{\noindent}\textbf{$\Lambda$CDM:} $\Lambda$ cold dark matter

{\noindent}\textbf{2dFGRS:} 2 degree Field Galactic Redshift Survey







































%%%%%%%%%%%%%%%%%%%%%%%%%%%%%%%%%%%%%%%%%%%%%%%%%%%%%%%%
%                                                      %
%                                                      %
%                     VARIABLES                        %
%                                                      %
%                                                      %
%%%%%%%%%%%%%%%%%%%%%%%%%%%%%%%%%%%%%%%%%%%%%%%%%%%%%%%%

\newpage
\subsection{Variables}

{\noindent}\textbf{Absolute magnitude}: $M ~ [{\rm mag}]$

{\noindent}\textbf{Absorption coefficient}: $\alpha_\nu ~ [{\rm cm^{-1}}]$

{\noindent}\textbf{Angstr\"{o}m}: \AA$ ~ [{\rm unit}]$

{\noindent}\textbf{Angular diameter distance}: $d_A ~ [{\rm pc}]$

{\noindent}\textbf{Angular size}: $\theta ~ [{\rm rad}]$

{\noindent}\textbf{Apparent magnitude}: $m ~ [{\rm mag}]$

{\noindent}\textbf{Baryon fraction}: $f_b ~ [{\rm dimensionless}]$

{\noindent}\textbf{Baryon temperature}: $T_b ~ [{\rm K}]$

{\noindent}\textbf{Binding energy of deuterium}: $E_\mathrm{D} ~ [{\rm MeV}]$

{\noindent}\textbf{Binding energy of helium}: $E_\mathrm{He} ~ [{\rm MeV}]$

{\noindent}\textbf{Binding energy of hydrogen}: $E_\mathrm{He} ~ [{\rm MeV}]$

{\noindent}\textbf{Black hole mass}: $M_\bullet ~ [{\rm M_\odot}]$

{\noindent}\textbf{Black hole radius of influence}: $r_\mathrm{BH} ~ [{\rm pc}]$

{\noindent}\textbf{Black hole radius of influence (angular)}: $r_\mathrm{BH} ~ [{\rm arcsec}]$

{\noindent}\textbf{Blackbody function}: $B_\nu(T) ~ [{\rm erg\,s^{-1}\,cm^{-2}\,Hz^{-1}\,sr^{-1}}]$

{\noindent}\textbf{Boltzmann constant}: $k_B ~ [{\rm m^2\,kg\,s^{-2}\,K^{-1}}]$

{\noindent}\textbf{Brightness temperature}: $T_b ~ [{\rm K}]$

{\noindent}\textbf{Brightness temperature of radio source}: $T_r ~ [{\rm K}]$

{\noindent}\textbf{Chandrasekhar mass}: $M_C ~ [{\rm M_\odot}]$

{\noindent}\textbf{CMB power spectrum amplitude}: $C_\ell ~ [{\rm units??}]$

{\noindent}\textbf{CMB temperature}: $T_\mathrm{CMB} ~ [{\rm K}]$

{\noindent}\textbf{CMB temperature fluctuations}: $\Delta T/T ~ [{\rm dimensionless}]$

{\noindent}\textbf{CNB temperature}: $T_\mathrm{CNB} ~ [{\rm K}]$

{\noindent}\textbf{Collisional coupling for species $i$}: $x_c^i ~ [{\rm
dimensionless}]$

{\noindent}\textbf{Collisional excitation rate}: $C_{10} ~ [{\rm s^{-1}}]$

{\noindent}\textbf{Comoving sound horizon}: $r_\mathrm{H,com} ~ [{\rm Mpc}]$

{\noindent}\textbf{Compton y-parameter}: $y ~ [{\rm dimensionless}]$

{\noindent}\textbf{Correlation function}: $C(\theta) ~ [{\rm dimensionless}]$

{\noindent}\textbf{Cosmological constant}: $\Lambda ~ [{\rm m^{-2}}]$

{\noindent}\textbf{Coupling coefficient of collisions}: $x_c ~ [{\rm dimensionless}]$

{\noindent}\textbf{Coupling coefficient of Ly$\alpha$ scattering}: $x_\alpha ~ [{\rm dimensionless}]$

{\noindent}\textbf{Coupling coefficient (total)}: $x_\mathrm{tot} ~ [{\rm dimensionless}]$

{\noindent}\textbf{Critical density}: $\rho_c ~ [{\rm g\,cm^{-3}}]$

{\noindent}\textbf{Cross section}: $\sigma ~ [{\rm m^{2}}]$

{\noindent}\textbf{Crossing timescale}: $t_\mathrm{cross} ~ [{\rm yr}]$

{\noindent}\textbf{Curvature term}: $S_\kappa(r) ~ [{\rm Mpc}]$

{\noindent}\textbf{Dark energy density}: $\rho_\Lambda ~ [{\rm g\,cm^{-3}}]$

{\noindent}\textbf{Deceleration parameter}: $q_0 ~ [{\rm dimensionless}]$

{\noindent}\textbf{Degeneracy}: \textit{see statistical weight}

{\noindent}\textbf{Density contrast}: $\delta(\mathbf{x},t) ~ [{\rm dimensionless}]$

{\noindent}\textbf{Density of baryons}: $\rho_b ~ [{\rm m^{-3}}]$

{\noindent}\textbf{Density of dark energy}: $\rho_\Lambda ~ [{\rm kg\,m^{-3}}]$

{\noindent}\textbf{Density of neutrinos}: $\rho_\nu ~ [{\rm m^{-3}}]$

{\noindent}\textbf{Density of neutrons}: $\rho_b ~ [{\rm m^{-3}}]$

{\noindent}\textbf{Density of protons}: $\rho_b ~ [{\rm m^{-3}}]$

{\noindent}\textbf{Density of vacuum}: $\rho_\mathrm{vac} ~ [{\rm eV\,m^{-3}}]$

{\noindent}\textbf{Density parameter}: $\Omega_i ~ [{\rm dimensionless}]$

{\noindent}\textbf{Density parameter of baryonic matter}: $\Omega_b ~
[{\rm dimensionless}]$

{\noindent}\textbf{Density parameter of curvature}: $\Omega_\kappa ~ [{\rm dimensionless}]$

{\noindent}\textbf{Density parameter of dark energy}: $\Omega_\Lambda ~ [{\rm dimensionless}]$

{\noindent}\textbf{Density parameter of dark matter}: $\Omega_c ~ [{\rm dimensionless}]$

{\noindent}\textbf{Density parameter of matter}: $\Omega_m ~ [{\rm dimensionless}]$

{\noindent}\textbf{Density parameter of neutrinos}: $\Omega_\nu ~ [{\rm dimensionless}]$

{\noindent}\textbf{Density parameter of radiation}: $\Omega_r ~ [{\rm dimensionless}]$

{\noindent}\textbf{Density perturbation field}: $\delta(\mathbf{x},t) ~ [{\rm dimensionless}]$

{\noindent}\textbf{Deuterium}: ${\rm D} ~ [{\rm element}]$

{\noindent}\textbf{Disk scale length}: $h_R ~ [{\rm Mpc}]$

{\noindent}\textbf{Distance}: $d ~ [{\rm m}]$

{\noindent}\textbf{Distance modulus}: $m - M ~ [{\rm mag}]$

{\noindent}\textbf{Effective number of neutrino families}: $N_\mathrm{eff} ~ [{\rm dimensionless}]$

{\noindent}\textbf{Effective radius}: $R_e ~ [{\rm Mpc}]$

{\noindent}\textbf{Einstein A coefficient (spontaneous emission)}: $A_{10} ~ [{\rm s^{-1}}]$

{\noindent}\textbf{Einstein B coefficient (spontaneous absorption)}: $B_{01} ~ [{\rm s^{-1}}]$

{\noindent}\textbf{Einstein B coefficient (stimulated emission)}: $B_{10} ~ [{\rm s^{-1}}]$

{\noindent}\textbf{Electron}: $e^- ~ [{\rm particle}]$

{\noindent}\textbf{Electronvolt}: ${\rm eV} ~ [{\rm unit}]$

{\noindent}\textbf{Electron optical depth}: $\tau_e ~ [{\rm dimensionless}]$

{\noindent}\textbf{Electroweak age}: $t_\mathrm{ew} ~ [{\rm s}]$

{\noindent}\textbf{Electroweak energy}: $E_\mathrm{ew} ~ [{\rm TeV}]$

{\noindent}\textbf{Electroweak temperature}: $T_\mathrm{ew} ~ [{\rm K}]$

{\noindent}\textbf{Electron radius (classical)}: $r_{e,0} ~ [{\rm cm}]$

{\noindent}\textbf{Elliptical $D_n$}: $D_n ~ [{\rm pc}]$

{\noindent}\textbf{Ellipticity}: $\epsilon ~ [{\rm dimensionless}]$

{\noindent}\textbf{Emission coefficient}: $j_\nu ~ [{\rm cm^{-1}}]$

{\noindent}\textbf{Energy}: $E ~ [{\rm eV\,or\,J}]$

{\noindent}\textbf{Energy density}: \textit{see density parameter}

{\noindent}\textbf{Energy of CMB photons}: $E_\mathrm{CMB} ~ [{\rm eV}]$

{\noindent}\textbf{Energy of photons}: $E_\gamma ~ [{\rm eV}]$

{\noindent}\textbf{Escape velocity}: $v)\mathrm{esp} ~ [{\rm km\,s^{-1}}]$

{\noindent}\textbf{Excitation temperature}: $T_{ex} ~ [{\rm K}]$

{\noindent}\textbf{Extinction coefficient}: $A_\nu ~ [{\rm mag}]$

{\noindent}\textbf{Extinction coefficient for the MWG}: $A_{V,\mathrm{MWG}} ~ [{\rm mag}]$

{\noindent}\textbf{Flux}: $F ~ [{\rm erg\,s^{-1}\,cm^{-2}}]$

{\noindent}\textbf{Flux (spectral)}: $F_\nu ~ [{\rm erg\,s^{-1}\,cm^{-2}\,Hz^{-1}}]$

{\noindent}\textbf{Frequency}: $\nu ~ [{\rm Hz}]$

{\noindent}\textbf{Fourier transform}: $P(k) ~ [{\rm function}]$

{\noindent}\textbf{Gas surface mass density}: $\Sigma_\mathrm{gas} ~ {\rm M_\odot\,pc^{-2}}$

{\noindent}\textbf{Gas temperature}: $T_g ~ [{\rm K}]$

{\noindent}\textbf{GUT age}: $t_\mathrm{GUT} ~ [{\rm s}]$

{\noindent}\textbf{GUT energy}: $E_\mathrm{GUT} ~ [{\rm TeV}]$

{\noindent}\textbf{GUT temperature}: $T_\mathrm{GUT} ~ [{\rm K}]$

{\noindent}\textbf{Gravitational constant}: $G ~ [{\rm m^3\,kg^{-1}\,s^{-2}}]$

{\noindent}\textbf{Growth factor}: $D_+(t) ~ [{\rm dimensionless}]$

{\noindent}\textbf{Gunn-Peterson optical depth}: $\tau_\mathrm{GP} ~ [{\rm dimensionless}]$

{\noindent}\textbf{Gunn-Peterson oscillator strength}: $f_\alpha ~ [{\rm dimensionless}]$

{\noindent}\textbf{Harrison-Zeldovich power spectrum}: $P(k) ~ [{\rm \mu K^2}]$

{\noindent}\textbf{Helium}: ${\rm He} ~ [{\rm element}]$

{\noindent}\textbf{Helium mass fraction}: $Y ~ [{\rm dimensionless}]$

{\noindent}\textbf{Horizon angular size}: $\theta_\mathrm{hor} ~ [{\rm rad}]$

{\noindent}\textbf{Horizon angular size (at recombination)}: $\theta_\mathrm{hor,rec} ~ [{\rm rad}]$

{\noindent}\textbf{Horizon distance}: $d_\mathrm{hor} ~ [{\rm Mpc}]$

{\noindent}\textbf{Hubble constant}: $H_0 ~ [{\rm km\,s^{-1}\,Mpc}]$

{\noindent}\textbf{Hubble distance}: $d_H ~ [{\rm Mpc}]$

{\noindent}\textbf{Hubble time}: $t_H ~ [{\rm Gyr}]$

{\noindent}\textbf{Hydrogen (atomic)}: ${\rm H} ~ [{\rm element}]$

{\noindent}\textbf{Hydrogen (molecular)}: ${\rm H_2} ~ [{\rm element}]$

{\noindent}\textbf{IMF}: $\xi(m) ~ [{\rm units???}]$

{\noindent}\textbf{Inflationary period}: ${\rm t_\mathrm{inflate}} ~ [{\rm s}]$

{\noindent}\textbf{Jeans mass}: $M_J ~ [{\rm M_\odot}]$

{\noindent}\textbf{Joule}: ${\rm J} ~ [{\rm unit}]$

{\noindent}\textbf{Kinetic energy}: $E_\mathrm{kin} ~ [{\rm J}]$

{\noindent}\textbf{Kinetic temperature}: $T_k ~ [{\rm K}]$

{\noindent}\textbf{Line profile}: $\phi(\nu) ~ [{\rm ???}]$

{\noindent}\textbf{Lithium}: ${\rm Li} ~ [{\rm element}]$

{\noindent}\textbf{Luminosity}: $L ~ [{\rm erg\,s^{-1}}]$

{\noindent}\textbf{Luminosity (Faber-Jackson)}: $L_\mathrm{FJ} ~ [{\rm erg\,s^{-1}}]$

{\noindent}\textbf{Luminosity (Tully-Fisher)}: $L_\mathrm{TF} ~ [{\rm erg\,s^{-1}}]$

{\noindent}\textbf{Luminosity distance}: $d_L ~ [{\rm Mpc}]$

{\noindent}\textbf{Luminosity of the Sun}: $L_\odot ~ [{\rm W}]$

{\noindent}\textbf{Luminosity of type Ia SN (at peak)}: $L_\mathrm{1a} ~ [{\rm L_\odot}]$

{\noindent}\textbf{Luminosity distance}: $d_L ~ [{\rm Mpc}]$

{\noindent}\textbf{Lyman-$\alpha$ rest wavelength}: $\lambda_\alpha$\,[\AA]

{\noindent}\textbf{Lyman-$\alpha$ temperature}: $T_\alpha ~ [{\rm K}]$

{\noindent}\textbf{Magnetic monopole density}: $n_M ~ [{\rm m^{-3}}]$

{\noindent}\textbf{Magnetic monopole mass}: $m_M ~ [{\rm kg}]$

{\noindent}\textbf{Magnitude (absolute)}: $M ~ [{\rm mag}]$

{\noindent}\textbf{Magnitude (apparent)}: $m ~ [{\rm mag}]$

{\noindent}\textbf{Mass}: $m\,{\rm or}\,M ~ [{\rm kg}]$

{\noindent}\textbf{Mass absorption coefficient}: $\kappa_\nu ~ [{\rm cm^2\,g^{-1}}]$

{\noindent}\textbf{Mass of the Sun}: $M_\odot ~ [{\rm kg}]$

{\noindent}\textbf{Mean free path}: $\ell ~ [{\rm m}]$

{\noindent}\textbf{Mean molecular weight}: $\mu ~ [{\rm dimensionless}]$

{\noindent}\textbf{Megaelectronvolt}: ${\rm MeV} ~ [{\rm unit}]$

{\noindent}\textbf{Metallicity}: $Z ~ [{\rm dimensionless}]$

{\noindent}\textbf{Multipole moment}: $C_\ell ~ [{\rm units???}]$

{\noindent}\textbf{Neutral fraction of hydrogen}: $x_\mathrm{HI} ~ [{\rm dimensionless}]$

{\noindent}\textbf{Neutron}: $n ~ [{\rm particle}]$

{\noindent}\textbf{Neutrino}: $\nu ~ [{\rm particle}]$

{\noindent}\textbf{Neutrino temperature}: $T_\nu ~ [{\rm K}]$

{\noindent}\textbf{Neutrino velocity}: $v_\nu ~ [{\rm km\,s^{-1}}]$

{\noindent}\textbf{Number density}: $n ~ [{\rm m^{-3}}]$

{\noindent}\textbf{Number density of baryons}: $n_b ~ [{\rm m^{-3}}]$

{\noindent}\textbf{Number density of electrons}: $n_e ~ [{\rm m^{-3}}]$

{\noindent}\textbf{Number density of Lyman-$\alpha$ forest lines}: $N_\alpha(z) ~ [{\rm dimensionless}]$

{\noindent}\textbf{Number density of neutrons}: $n_n ~ [{\rm m^{-3}}]$

{\noindent}\textbf{Number density of protons}: $n_p ~ [{\rm m^{-3}}]$

{\noindent}\textbf{Optical depth}: $\tau ~ [{\rm dimensionless}]$

{\noindent}\textbf{Parallax}: $p ~ [{\rm rad~or~^\circ}]$

{\noindent}\textbf{Parsec}: ${\rm pc} ~ [{\rm unit}]$

{\noindent}\textbf{Peculiar velocity}: $v_{\rm pec} ~ [{\rm km\,s^{-1}}]$

{\noindent}\textbf{Period}: $P ~ [{\rm yr}]$

{\noindent}\textbf{Photon}: $\gamma ~ [{\rm particle}]$

{\noindent}\textbf{Photon temperature}: $T_\gamma ~ [{\rm K}]$

{\noindent}\textbf{Photon scattering rate (ionized)}: $\Gamma ~ [{\rm s^{-1}}]$

{\noindent}\textbf{Planck constant}: $h ~ [{\rm m^2\,kg\,s^{-1}}]$

{\noindent}\textbf{Planck constant (reduced)}: $\hbar ~ [{\rm m^2\,kg\,s^{-1}}]$

{\noindent}\textbf{Planck energy}: $E_P ~ [{\rm eV}]$

{\noindent}\textbf{Planck function}: $B_\nu(T) ~ [{\rm erg\,s^{-1}\,cm^{-2}\,Hz^{-1}\,sr^{-1}}]$

{\noindent}\textbf{Planck length}: $\ell_P ~ [{\rm m}]$

{\noindent}\textbf{Planck mass}: $m_P ~ [{\rm kg}]$

{\noindent}\textbf{Planck time}: $t_P ~ [{\rm s}]$

{\noindent}\textbf{Polarized angle}: $\chi ~ [{\rm rad}]$

{\noindent}\textbf{Polarized intensity}: $P ~ [{\rm Jy}]$

{\noindent}\textbf{Positron}: $e^+ ~ [{\rm particle}]$

{\noindent}\textbf{Potential energy}: $E_\mathrm{pot} ~ [{\rm J}]$

{\noindent}\textbf{Power spectrum}: $P(k) ~ [{\rm units???}]$

{\noindent}\textbf{Proper distance}: $d_p ~ [{\rm Mpc}]$

{\noindent}\textbf{Proper motion}: $\mu ~ [{\rm ''\,year^{-1}}]$

{\noindent}\textbf{Proton}: $p ~ [{\rm particle}]$

{\noindent}\textbf{Radius of curvature}: $R_0 ~ [{\rm Mpc}]$

{\noindent}\textbf{Radiative flux}: $F ~ [{\rm W\,m^{-2}}]$

{\noindent}\textbf{Redshift}: $z ~ [{\rm dimensionless}]$

{\noindent}\textbf{Redshift (cosmological)}: $z_\mathrm{cos} ~ [{\rm dimensionless}]$

{\noindent}\textbf{Redshift of last scattering}: $z_\mathrm{ls} ~ [{\rm dimensionless}]$

{\noindent}\textbf{Redshift of matter-$\Lambda$ equality}: $z_{m\Lambda} ~ [{\rm dimensionless}]$

{\noindent}\textbf{Redshift of radiation-matter equality}: $z_{rm} ~ [{\rm dimensionless}]$

{\noindent}\textbf{Redshift of thermal coupling}: $z_t ~ [{\rm dimensionless}]$

{\noindent}\textbf{Reduced Hubble constant}: $h_0 ~ [{\rm km\,s^{-1}\,Mpc}]$

{\noindent}\textbf{Reduced Planck constant}: $\hbar ~ [{\rm m^2\,kg\,s^{-1}}]$

{\noindent}\textbf{Relaxation time}: $t_\mathrm{relax} ~ [{\rm yr}]$

{\noindent}\textbf{Rest mass of electron}: $m_e ~ [{\rm kg}]$

{\noindent}\textbf{Rest mass of neutron}: $m_n ~ [{\rm kg}]$

{\noindent}\textbf{Rest mass of proton}: $m_p ~ [{\rm kg}]$

{\noindent}\textbf{Rotational velocity (at Solar position)}: $V_0 ~ [{\rm km\,s^{-1}}]$

{\noindent}\textbf{Scalar spectral index}: $n_s ~ [{\rm dimensionless}]$

{\noindent}\textbf{Scale factor}: $a ~ [{\rm dimensionless}]$

{\noindent}\textbf{Scale height}: $h_z ~ [{\rm pc}]$

{\noindent}\textbf{Scattering rate between hydrogen atoms}: $k_{1-0}^\mathrm{HH} ~ [{\rm s^{-1}}]$

{\noindent}\textbf{Scattering rate between hydrogen atoms and electrons}: $k_{1-0}^{e\mathrm{H}} ~ [{\rm s^{-1}}]$

{\noindent}\textbf{Scattering rate between hydrogen atoms and protons}: $k_{1-0}^{p\mathrm{H}} ~ [{\rm s^{-1}}]$

{\noindent}\textbf{Schwarzschild radius}: $r_S ~ [{\rm pc}]$

{\noindent}\textbf{Semi-major axis}: $a ~ [{\rm Mpc}]$

{\noindent}\textbf{Semi-minor axis}: $b ~ [{\rm Mpc}]$

{\noindent}\textbf{Sign of curvature}: $\kappa ~ [{\rm dimensionless}]$

{\noindent}\textbf{Silk mass}: $M_s ~ [{\rm kg}]$

{\noindent}\textbf{Solid angle}: $\Omega ~ [{\rm sr}]$

{\noindent}\textbf{Sound speed}: $v_s ~ [{\rm m\,s^{-1}}]$

{\noindent}\textbf{Spherical harmonics}: $Y_{lm}(\theta,\phi) ~ [{\rm dimensionless}]$

{\noindent}\textbf{Spin temperature}: $T_s ~ [{\rm K}]$

{\noindent}\textbf{Specific intensity}: $I_\nu ~ [{\rm erg\,s^{-1}\,cm^{-2}\,Hz^{-1}\,sr^{-1}}]$

{\noindent}\textbf{Speed of light}: $c ~ [{\rm m\,s^{-1}}]$

{\noindent}\textbf{Spin de-excitation rate via collisions for species $i$}: $k_{10}^i ~ [{\rm cm^3s^{-1}}]$

{\noindent}\textbf{Spin temperature}: $T_s ~ [{\rm K}]$

{\noindent}\textbf{Star formation history}: $\psi(z) ~ [{\rm M_\odot\,year^{-1}\,Mpc^{-3}}]$

{\noindent}\textbf{Star formation rate per unit area}: $\Sigma_\mathrm{SFR} ~ [{\rm M_\odot\,year^{-1}\,pc^{-2}}]$

{\noindent}\textbf{Statistical weight}: $g ~ [{\rm dimensionless}]$

{\noindent}\textbf{Stefan-Boltzmann constant}: $\sigma ~ [{\rm W\,m^{-2}\,K^{-4}}]$

{\noindent}\textbf{Steradian}: ${\rm sr} ~ [{\rm unit}]$

{\noindent}\textbf{Stokes Q}: $Q ~ [{\rm Jy}]$

{\noindent}\textbf{Stokes U}: $Q ~ [{\rm Jy}]$

{\noindent}\textbf{Stress energy tensor}: $T_{\alpha\beta} ~ [{\rm unit}]$

{\noindent}\textbf{Sunyaev-Zeldovich temperature fluctuation}: $\Delta T_\mathrm{SZ} ~ [{\rm K}]$

{\noindent}\textbf{Sum of neutrino masses}: $\Sigma m_\nu ~ [{\rm eV}]$

{\noindent}\textbf{Surface brightness}: $\mu ~ [{\rm mag\,arcsec^{-2}}]$

{\noindent}\textbf{Surface brightness (central)}: $\mu_0 ~ [{\rm mag\,arcsec^{-2}}]$

{\noindent}\textbf{Surface brightness at effective radius $R_e$}: $\mu_e ~ [{\rm mag\,arcsec^{-2}}]$

{\noindent}\textbf{Temperature}: $T ~ [{\rm K}]$

{\noindent}\textbf{Temperature of the CMB}: $T_{\rm CMB} ~ [{\rm K}]$

{\noindent}\textbf{Temperature of recombination}: $T_{\rm rec} ~ [{\rm K}]$

{\noindent}\textbf{Temperature of the Sun}: $T_\odot ~ [{\rm K}]$

{\noindent}\textbf{Thermal energy}: $E_\mathrm{th} ~ [{\rm J}]$

{\noindent}\textbf{Thomson cross section}: $\sigma_T ~ [{\rm m^2}]$

{\noindent}\textbf{Transfer function}: $T(k) ~ [{\rm ???}]$

{\noindent}\textbf{Tritium}: $T ~ [{\rm element}]$

{\noindent}\textbf{Two-point correlation function:} $\xi(x) ~ [{\rm dimensionless}]$

{\noindent}\textbf{Velocity}: $v ~ [{\rm m\,s^{-1}}]$

{\noindent}\textbf{Velocity (circular)}: $v_c ~ [{\rm m\,s^{-1}}]$

{\noindent}\textbf{Velocity (radial)}: $v_r ~ [{\rm m\,s^{-1}}]$

{\noindent}\textbf{Velocity (rotational)}: $v_\mathrm{rot} ~ [{\rm m\,s^{-1}}]$

{\noindent}\textbf{Velocity (tangential)}: $v_t ~ [{\rm m\,s^{-1}}]$

{\noindent}\textbf{Velocity dispersion}: $\sigma_v ~ [{\rm m\,s^{-1}}]$

{\noindent}\textbf{Virial temperature}: $T_\mathrm{vir} ~ [{\rm K}]$

{\noindent}\textbf{Wavelength}: $\lambda ~ [{\rm m}]$

{\noindent}\textbf{Wavenumber (comoving)}: $k ~ [{\rm m^{-1}}]$





































%%%%%%%%%%%%%%%%%%%%%%%%%%%%%%%%%%%%%%%%%%%%%%%%%%%%%%%%
%                                                      %
%                                                      %
%                   USEFUL VALUES                      %
%                                                      %
%                                                      %
%%%%%%%%%%%%%%%%%%%%%%%%%%%%%%%%%%%%%%%%%%%%%%%%%%%%%%%%

\newpage
\subsection{Useful Values}

{\noindent}\textbf{Age of the Earth}: $t_\oplus = 4.5\,[{\rm Gyr}]$

{\noindent}\textbf{Age of the Solar System}: $t_\mathrm{SS} = 4.6\,[{\rm Gyr}]$

{\noindent}\textbf{Age of the Universe}: $t_\mathrm{universe} = 13.7\,[{\rm Gyr}]$

{\noindent}\textbf{Angstr\"{o}m}: \AA$ = 10^{-10}\,[{\rm m}]$

{\noindent}\textbf{Astronomical Unit}: $1$ AU = $1.5\times10^{11}\,[{\rm m}]$

{\noindent}\textbf{Binding energy of deuterium}: $E_\mathrm{D} = 1.1 ~ [{\rm MeV}]$

{\noindent}\textbf{Binding energy of helium}: $E_\mathrm{He} = 7 ~ [{\rm MeV}]$

{\noindent}\textbf{Binding energy of hydrogen}: $E_\mathrm{He} 13.7 ~ [{\rm MeV}]$

{\noindent}\textbf{Chandrasekhar mass}: $M_C = 1.4 ~ [{\rm M_\odot}]$

{\noindent}\textbf{CMB peak photon wavelength}: $\lambda_\mathrm{CMB} = 2\,[{\rm mm}]$

{\noindent}\textbf{CMB temperature}: $T_\mathrm{CMB} = 2.725\,[{\rm K}]$

{\noindent}\textbf{CMB temperature fluctuations}: $\Delta T/T\approx10^{-5} ~ [{\rm dimensionless}]$

{\noindent}\textbf{CNB temperature}: $T_\mathrm{CNB} = 1.9\,[{\rm K}]$

{\noindent}\textbf{Cosmological constant}: $\Lambda=1.11\times10^{-52} ~ [{\rm m^{-2}}]$

{\noindent}\textbf{Critical density (current)}: $\rho_{c,0}\approx10^{-26}\,[{\rm kg\,m^{-3}}]$

{\noindent}\textbf{Density of dark energy}: $\rho_\Lambda = 10^{-27} ~ [{\rm kg\,m^{-3}}]$

{\noindent}\textbf{Density parameter of baryons}: $\Omega_b \approx 0.044 ~ [{\rm dimensionless}]$

{\noindent}\textbf{Density parameter of CMB}: $\Omega_\mathrm{CMB} \approx 5\times10^{-5} ~ [{\rm dimensionless}]$

{\noindent}\textbf{Density parameter of curvature}: $\Omega_\kappa \approx 0.0 ~ [{\rm dimensionless}]$

{\noindent}\textbf{Density parameter of cold dark matter}: $\Omega_c \approx 0.23 ~ [{\rm dimensionless}]$

{\noindent}\textbf{Density parameter of dark energy}: $\Omega_\Lambda \approx 0.73 ~ [{\rm dimensionless}]$

{\noindent}\textbf{Density parameter of matter}: $\Omega_m \approx 0.27 ~ [{\rm dimensionless}]$

{\noindent}\textbf{Density parameter of neutrinos}: $\Omega_\nu < 0.12 ~ [{\rm dimensionless}]$

{\noindent}\textbf{Density parameter of radiation}: $\Omega_r \approx 5\times10^{-5} ~ [{\rm dimensionless}]$

{\noindent}\textbf{Dipole}: $\ell=1 ~ [{\rm dimensionless}]$

{\noindent}\textbf{Electronvolt in Joules}: $1\,{\rm eV} = 1.6\times10^{-19}\,{\rm J}$

{\noindent}\textbf{Electron radius (classical)}: $r_{e,0}=2.82\times10^{-13} ~ [{\rm cm}]$

{\noindent}\textbf{Electroweak age}: $t_\mathrm{ew} \sim 10^{-12}\,[{\rm s}]$

{\noindent}\textbf{Electroweak energy}: $E_\mathrm{ew} \sim 1\,[{\rm TeV}]$

{\noindent}\textbf{Electroweak temperature}: $T_\mathrm{ew} \sim 10^{16}\,[{\rm K}]$

{\noindent}\textbf{Energy of CMB photons}: $E_{\gamma,\mathrm{CMB}} \sim 6\times10^{-4}\,[{\rm eV}]$

{\noindent}\textbf{Energy of nuclei fission}: $E_\mathrm{fission} \sim 1\,[{\rm MeV}]$

{\noindent}\textbf{Energy of nuclei ionization}: $E_\mathrm{ion} \sim 10\,[{\rm eV}]$

{\noindent}\textbf{GUT age}: $t_\mathrm{GUT} \sim 10^{-36}\,[{\rm s}]$

{\noindent}\textbf{GUT energy}: $E_\mathrm{GUT} \sim 10^{12}-10^{13}\,[{\rm TeV}]$

{\noindent}\textbf{GUT temperature}: $T_\mathrm{GUT} \sim 10^{28}\,[{\rm K}]$

{\noindent}\textbf{Helium mass fraction}: $Y = 0.25 ~ [{\rm dimensionless}]$

{\noindent}\textbf{Hubble constant}: $H_0 = 70\,[{\rm km\,s^{-1}\,Mpc^{-1}}]$

{\noindent}\textbf{Hubble constant (original)}: $H_0 \approx 500\,[{\rm km\,s^{-1}\,Mpc^{-1}}]$

{\noindent}\textbf{Hubble distance}: $d_H \sim 0.2\,[{\rm Mpc}]$

{\noindent}\textbf{Hubble time}: $t_H \sim 14\,[{\rm Gyr}]$

{\noindent}\textbf{IMF normalization}: $\int\limits_{m_L}^{m_U} m\xi(m)\,\mathrm{d}m = 1\,{\rm M}_\odot$

{\noindent}\textbf{Inflationary period}: $t_\mathrm{inflate} = 10^{-35}\,[{\rm s}]$

{\noindent}\textbf{Ionization energy of atomic hydrogen}: $E_\mathrm{ion}^\mathrm{H} = 13.6\,[{\rm eV}]$

{\noindent}\textbf{Ionization energy of atomic helium}: $E_\mathrm{ion}^\mathrm{He} = 24.6\,[{\rm eV}]$

{\noindent}\textbf{Ionization energy of molecular hydrogen}: $E_\mathrm{ion}^\mathrm{H_2} = 11.26\,[{\rm eV}]$

{\noindent}\textbf{Large cosmological scale}: $\sim 100\,[{\rm Mpc}]$

{\noindent}\textbf{Legendre Polynomials}: $P_l$

{\noindent}\textbf{Line profile normalization}: $\int\limits_0^\infty \phi(\nu)=1 ~ [{\rm ???}]$

{\noindent}\textbf{Luminosity of the Sun}: $L_\odot=3.8\times10^6 ~ [{\rm W}]$

{\noindent}\textbf{Luminosity of type Ia SN (at peak)}: $L_\mathrm{1a}=4\times10^9 ~ [{\rm L_\odot}]$

{\noindent}\textbf{Lyman-alpha rest wavelength}: $\lambda_\alpha = 1216$\,[\AA]

{\noindent}\textbf{Magnitude (absolute) of the Sun}: $M = 26.8 ~ [{\rm dimensionless}]$

{\noindent}\textbf{Magnitude (apparent) of the Sun}: $m = 4.74 ~ [{\rm dimensionless}]$

{\noindent}\textbf{Mass of an electron}: $m_e = 9.109\times10^{-31}\,[{\rm kg}]$

{\noindent}\textbf{Mass of a neutron}: $m_n = 1.67\times10^{-27}\,[{\rm kg}]$

{\noindent}\textbf{Mass of a proton}: $m_p = 1.67\times10^{-27}\,[{\rm kg}]$

{\noindent}\textbf{Megaelectronvolt}: ${\rm MeV} = 10^{10.065}\,[{\rm K}]$

{\noindent}\textbf{Metallicity of the Sun}: $Z\approx0.02 ~ [{\rm dimensionless}]$

{\noindent}\textbf{Milky Way luminosity}: ${\rm L_{MWG}} = 3.6\times10^{10}\,[{\rm L}_\odot]$

{\noindent}\textbf{Monopole}: $\ell=0 ~ [{\rm dimensionless}]$

{\noindent}\textbf{Neutron half life}: $t_n\sim\,890\,[{\rm sec}]\,{\rm or}\,\sim\,15\,[{\rm min}]$

{\noindent}\textbf{Neutron-proton freezeout}: $\frac{n_n}{n_p} \simeq \frac{1}{7}\,[{\rm dimensionless}]$

{\noindent}\textbf{Neutron-proton rest mass-energy difference}: $(m_n-m_p)c^2 = 1.3\,[{\rm MeV}]$

{\noindent}\textbf{Number density of baryons}: $n_{b,0} = 0.22\,[{\rm m^{-3}}]$

{\noindent}\textbf{Number density of CMB photons}: $n_\gamma= 4.11\times10^8\,[{\rm m^{-3}}]$

{\noindent}\textbf{Number density ratio of baryons to CMB photons}: $\eta \equiv \frac{n_{b,0}}{n_\gamma} = 5\times10^{-10}\,[{\rm dimensionless}]$

{\noindent}\textbf{Parsec}: $1\,{\rm pc} = 206265\,{\rm AU} = 3.1\times10^{16}\,{\rm m}$

{\noindent}\textbf{Photoionization energy of hydrogen}: $E^\mathrm{H}_\mathrm{ion} = 13.6\,[{\rm eV}]$

{\noindent}\textbf{Planck energy}: $E_P = 10^{16}\,[{\rm TeV}]$

{\noindent}\textbf{Planck length}: $\ell_P = 1.6\times10^{-35}\,[{\rm m}]$

{\noindent}\textbf{Planck mass}: $M_P = 2.2\times10^{-8}\,[{\rm kg}]$

{\noindent}\textbf{Planck temperature}: $T_P = 1.4\times10^{32}\,[{\rm K}]$

{\noindent}\textbf{Planck time}: $t_P = 5.4\times10^{-44}\,[{\rm s}]$

{\noindent}\textbf{Proper surface area}: $A_p(t_0) ~ [{\rm Mpc^2}]$

{\noindent}\textbf{Recombination age of Universe}: $t_\mathrm{rec} \sim 380,000\,[{\rm yr}]$

{\noindent}\textbf{Recombination redshift}: $z_\mathrm{rec}\sim1,100\,[{\rm dimensionless}]$

{\noindent}\textbf{Reduced Hubble constant}: $\dfrac{H_0}{100} = 0.70\,[{\rm km\,s^{-1}\,Mpc^{-1}}]$

{\noindent}\textbf{Redshift of radiation-matter equality}: $z_{rm} \approx 5399\,[{\rm dimensionless}]$

{\noindent}\textbf{Redshift of matter-$\Lambda$ equality}: $z_{m\Lambda}
\approx0.39\,[{\rm dimensionless}]$

{\noindent}\textbf{Rest energy of an deuterium}: $m_\mathrm{D}c^2 = 2.225\,[{\rm MeV}]$

{\noindent}\textbf{Rest energy of an electron}: $m_ec^2 = 0.511\,[{\rm MeV}]$

{\noindent}\textbf{Rest energy of a neutron}: $m_nc^2 = 939.6\,[{\rm MeV}]$

{\noindent}\textbf{Rest energy of a proton}: $m_pc^2 = 938.3\,[{\rm MeV}]$

{\noindent}\textbf{Rotational velocity (at Solar position)}: $V_0 = 220 ~ [{\rm km\,s^{-1}}]$

{\noindent}\textbf{Scalar spectral index}: $n_s\approx1 ~ [{\rm dimensionless}]$

{\noindent}\textbf{Scale factor (current)}: $a_0=a(t_0)\equiv1\,[{\rm dimensionless}]$

{\noindent}\textbf{Scalefactor of radiation-matter equality}: $a_{rm} \approx 2.8\times10^{-4}\,[{\rm dimensionless}]$

{\noindent}\textbf{Scalefactor of matter-$\Lambda$ equality}: $a_{m\Lambda}
\approx0.75\,[{\rm dimensionless}]$

{\noindent}\textbf{Seconds in a year}: $1\,{\rm yr} = 3.2\times10^7\,[{\rm s}]$

{\noindent}\textbf{Seconds in a gigayear}: $1\,{\rm Gyr} = 3.2\times10^{16}\,[{\rm s}]$

{\noindent}\textbf{Sign of curvature (closed Universe)}: $\kappa>0 ~ [{\rm dimensionless}]$

{\noindent}\textbf{Sign of curvature (flat Universe)}: $\kappa=0 ~ [{\rm dimensionless}]$

{\noindent}\textbf{Sign of curvature (open Universe)}: $\kappa<0 ~ [{\rm dimensionless}]$

{\noindent}\textbf{Solar absolute magnitude}: ${\rm M}_\odot = 4.74\,[{\rm mag}]$

{\noindent}\textbf{Solar apparent magnitude}: ${\rm m}_\odot = 26.8\,[{\rm mag}]$

{\noindent}\textbf{Solar flux}: ${\rm F}_\odot = 1367\,[{\rm W\,m^{-2}}]$

{\noindent}\textbf{Solar luminosity}: ${\rm L}_\odot = 3.8\times10^{26}\,[{\rm W}]$

{\noindent}\textbf{Solar mass}: ${\rm M}_\odot = 2.0\times10^{30}\,[{\rm kg}]$

{\noindent}\textbf{Solar mean photon energy}: $\langle E_\odot \rangle = 1.3\,[{\rm eV}]$

{\noindent}\textbf{Solar temperature}: $T_\odot = 5800\,[{\rm K}]$

{\noindent}\textbf{Speed of light}: $c=2.998\times10^8\,[{\rm m\,s^{-1}}]$

{\noindent}\textbf{Stefan-Boltzmann constant}: $\sigma=5.67\times10^{-8} ~ [{\rm W\,m^{-2}\,K^{-4}}]$

{\noindent}\textbf{Supernova peak luminosity}: $L_\mathrm{SN,max} = 4\times10^9\,{\rm L_\odot}\,[{\rm W}]$

{\noindent}\textbf{Temperature at CMB decoupling}: $k_BT_\mathrm{CMB,dec} = 1/4\,[{\rm eV}]$

{\noindent}\textbf{Thomson cross section}: $\sigma_T=0.665\times10^{-24} ~ [{\rm cm^2}]$

{\noindent}\textbf{Time of radiation-matter equality}: $t_{rm} \approx 4.7\times10^{4}\,[{\rm yr}]$

{\noindent}\textbf{Time of matter-$\Lambda$ equality}: $t_{m\Lambda}
\approx9.8\,[{\rm Gyr}]$

{\noindent}\textbf{Vacuum energy density}: $\rho_\mathrm{vac} = 10^{133} ~ [{\rm eV\,cm^{-3}}]$

{\noindent}\textbf{21 cm energy}: $E_{21\,\mathrm{cm}} = 5.9\times10^{-6} ~ [{\rm eV}]$

{\noindent}\textbf{21 cm rest frequency}: $\nu_{21\,\mathrm{cm}} = 1420 ~ [{\rm MHz}]$






































%%%%%%%%%%%%%%%%%%%%%%%%%%%%%%%%%%%%%%%%%%%%%%%%%%%%%%%%
%                                                      %
%                                                      %
%                    EQUATIONS                         %
%                                                      %
%                                                      %
%%%%%%%%%%%%%%%%%%%%%%%%%%%%%%%%%%%%%%%%%%%%%%%%%%%%%%%%

\newpage
\subsection{Equations}

{\noindent}\textbf{Absolute magnitude}:
\begin{align*}
    M &\equiv -2.5\log\left(\frac{L}{L_0}\right) ~ [{\rm mag}] \\
    M &\equiv -2.5\log\left(\frac{L}{78.7\,{\rm L_\odot}}\right) ~ [{\rm mag}]
\end{align*}

{\noindent}\textbf{Absorption coefficient}: $\alpha_\nu ~ [{\rm cm^{-1}}]$
\begin{align*}
    \alpha_\nu = n\sigma_\nu = \rho\kappa_\nu ~ [{\rm cm^{-1}}]
\end{align*}

{\noindent}\textbf{Angular diameter distance}:
\begin{align*}
    D_A = D(1+z)^{-1} ~ [{\rm pc}]
\end{align*}

{\noindent}\textbf{Apparent magnitude}:
\begin{align*}
     m &\equiv -2.5\log\left(\frac{F}{F_0}\right) ~ [{\rm mag}] \\
     m &\equiv -2.5\log\left(\frac{F}{2.53\times10^{-8}\,{\rm W\,m^{-2}}}\right) ~ [{\rm mag}]
\end{align*}

{\noindent}\textbf{Apparent-absolute magnitude relation}:
\begin{align*}
    M &= m -5\log_{10}\left(\frac{d_L}{10\,{\rm pc}}\right) ~ [{\rm mag}] \\
      &= m -5\log_{10}\left(\frac{d_L}{1\,{\rm Mpc}}\right)-25 ~ [{\rm mag}]
\end{align*}

{\noindent}\textbf{Baryon fraction}:
\begin{align*}
    f_b \equiv \frac{\Omega_b}{\Omega_m} ~ [{\rm dimensionless}]
\end{align*}

{\noindent}\textbf{Black hole mass-luminosity relation}:
\begin{align*}
    M_\bullet = 1.7\times10^9\left(\frac{L_\mathrm{V}}{10^{11}\,{\rm L_{V_\odot}}}\right)^{1.11} ~ [{\rm M_\odot}]
\end{align*}

{\noindent}\textbf{Black hole mass-bulge relation}:
\begin{align*}
    M_\bullet &= 2.9\times10^8\left(\frac{M_\mathrm{bulge}}{10^11\,{\rm M_\odot}}\right)^{1.05} ~ [{\rm M_\odot}] \\
    M_\bullet &\approx 3\times10^{-3}M_\mathrm{bulge}
\end{align*}

{\noindent}\textbf{Black hole mass-velocity dispersion relation}:
\begin{align*}
    M_\bullet = 2.1\times 10^8\left(\frac{\sigma_v}{200\,{\rm km\,s^{-1}}}\right)^{5.64} ~ [{\rm M_\odot}]
\end{align*}

{\noindent}\textbf{Black hole radius of influence}:
\begin{align*}
    r_\mathrm{BH} = \frac{GM_\bullet}{\sigma_v^2} \sim 0.4\left(\frac{M_\bullet}{10^6\,{\rm M_\odot}}\right)\left(\frac{\sigma_v}{100\,{\rm km\,s^{-1}}}\right)^{-2} ~ [{\rm pc}]
\end{align*}

{\noindent}\textbf{Black hole radius of influence (angular)}:
\begin{align*}
    \theta_\mathrm{BH} = \frac{r_\mathrm{BH}}{d} \sim 0.1\left(\frac{M_\bullet}{10^6\,{\rm M_\odot}}\right) \left(\frac{\sigma_v}{100\,{\rm km\,s^{-1}}}\right)^{-2} \left(\frac{d}{1\,{\rm Mpc}}\right)^{-1} ~ [{\rm arcsec}]
\end{align*}

{\noindent}\textbf{Blackbody function}:
\begin{align*}
    B_\nu(T) = \frac{2h\nu^2}{c^2} \frac{1}{\exp(h\nu/k_BT) - 1} ~ [{\rm erg\,s^{-1}\,cm^{-2}\,Hz^{-1}\,sr^{-1}}]
\end{align*}

{\noindent}\textbf{Blackbody mean photon energy}:
\begin{align*}
    \langle E_\mathrm{bb} \rangle = 2.7\,k_BT  ~ [{\rm eV}]
\end{align*}

{\noindent}\textbf{Blackbody peak photon energy}:
\begin{align*}
    E_\mathrm{bb_{peak}} = 2.82\,k_BT ~ [{\rm eV}]
\end{align*}

{\noindent}\textbf{Boltzmann factor}:
\begin{align*}
    \frac{n_i}{n_j} = \mathrm{e}^{-\Delta mc^2/k_BT} ~ [{\rm dimensionless}]
\end{align*}

{\noindent}\textbf{CMB temperature fluctuations}:
\begin{align*}
    \frac{\delta T}{T}(\theta,\phi) = \sum\limits_{\ell=0}^{\infty}\sum\limits_{m=-1}^{\ell} a_{\ell m}Y_{\ell m}(\theta,\phi) ~ [{\rm dimensionless}]
\end{align*}

{\noindent}\textbf{Comoving sound horizon}:
\begin{align*}
    r_\mathrm{H,com}(z) &= \int\limits_0^t \frac{c\mathrm{d}t}{a(t)} ~ [{\rm Mpc}]
    &= \int\limits_0^{(1+z)^{-1}} \frac{c\mathrm{d}a}{a^2H(a)} ~ [{\rm Mpc}]
\end{align*}

{\noindent}\textbf{Cosmological constant}:
\begin{align*}
    \Lambda = \frac{3H_0^2}{c^2}\Omega_\Lambda ~ [{\rm m^{-2}}]
\end{align*}

{\noindent}\textbf{Correlation function of CMB temperature anisotropies}:
\begin{align*}
    C(\theta) = \left\langle\frac{\delta T}{T}(\hat{n})\frac{\delta T}{T}(\hat{n}')\right\rangle_{\hat{n}\cdot\hat{n}'=\cos\theta} \\
    C(\theta) = \frac{1}{4\pi} \sum\limits_{\ell=0}^\infty (2\ell+\ell)(C_\ell)P_\ell(cos\theta),
\end{align*}

{\noindent}\textbf{Collisional coupling coefficient}:
\begin{align*}
    x_c &= x_c^\mathrm{HH} + x_c^\mathrm{eH} + x_c^\mathrm{pH} \\
        &= \frac{T_*}{A_{10}T_\gamma} \left[k_{1-0}^\mathrm{HH}(T_k)n_\mathrm{H}+k_{1-0}^{e\mathrm{H}}(T_k)n_e+k_{1-0}^{p\mathrm{H}}(T_k)n_p\right] ~ [{\rm dimensionless}]
\end{align*}

{\noindent}\textbf{Coupling coefficient (total)}:
\begin{align*}
    x_\mathrm{tot} \equiv x_c + x_\alpha ~ [{\rm dimensionless}]
\end{align*}

{\noindent}\textbf{Critical density}:
\begin{align*}
    \rho_c \equiv \dfrac{3H_0^2}{8\pi G}  ~ [{\rm g\,cm^{-3}}]
\end{align*}

{\noindent}\textbf{Crossing timescale}:
\begin{align*}
    t_\mathrm{cross} = \frac{R}{v} ~ [{\rm yr}
\end{align*}

{\noindent}\textbf{Curvature term}:
\begin{align*}
S_\kappa(r) =
\left\{
\begin{aligned}
R\sin(r/R), ~~~~~& (\kappa = +1) \\
          r,~~~~~& (\kappa = 0) \\
R\sinh(r/R),~~~~~& (\kappa = -1)
\end{aligned}
\right.
\end{align*}

{\noindent}\textbf{Dark energy density}:
\begin{align*}
    \rho_\Lambda \equiv \dfrac{\Lambda}{8\pi G} ~ [{\rm g\,cm^{-3}}]
\end{align*}

{\noindent}\textbf{Deceleration parameter}:
\begin{align*}
    q_0 = \Omega_{r,0} + \frac{1}{2}\Omega_{m,0} - \Omega_\Lambda ~ [{\rm dimensionless}]
\end{align*}

{\noindent}\textbf{Density contrast}:
\begin{align*}
    \delta(\mathbf{x},t) \equiv \frac{\rho(\mathbf{x},t)-\langle\rho(t)\rangle}{\langle\rho(t)\rangle} ~ [{\rm dimensionless}]
\end{align*}

{\noindent}\textbf{Density parameter}:
\begin{align*}
    \Omega_i \equiv \frac{\rho_i}{\rho_c} ~ [{\rm dimensionless}]
\end{align*}

{\noindent}\textbf{Density parameter (total)}:
\begin{align*}
    \Omega = \sum\Omega_i \equiv \sum \frac{\rho_i}{\rho_c} ~ [{\rm dimensionless}]
\end{align*}

{\noindent}\textbf{Density parameter of curvature}:
\begin{align*}
    \Omega_\kappa = 1-\Omega_0 ~ [{\rm dimensionless}]
\end{align*}

{\noindent}\textbf{Density parameter of dark energy}:
\begin{align*}
    \Omega_\Lambda &\equiv \dfrac{\rho_\Lambda}{\rho_c} \\
    &= \left(\dfrac{\Lambda}{\cancel{8\pi G}}\right) \left(\dfrac{\cancel{8\pi G}}{3H_0}\right) \\
    &= \dfrac{\Lambda}{3H_0} ~ [{\rm dimensionless}] ~ (???)
\end{align*}

{\noindent}\textbf{Density parameter of matter}:
\begin{align*}
    \Omega_m(a) \equiv \dfrac{\rho_m}{\rho_c}a(t)^{-3} ~ [{\rm dimensionless}]
\end{align*}

{\noindent}\textbf{Density parameter of radiation}:
\begin{align*}
    \Omega_r(a) \equiv \dfrac{\rho_r}{\rho_c}a(t)^{-4} ~ [{\rm dimensionless}]
\end{align*}

{\noindent}\textbf{Density-redshift-scalefactor relation}:
\begin{align*}
    \bar{\rho}(z) = \bar{\rho}_0(1+z)^3 = \frac{\bar{\rho}_0}{a^3} ~ [{\rm g\,cm^{-3}}]
\end{align*}

{\noindent}\textbf{Distance (moving cluster)}:
\begin{align*}
    d = \frac{v_r\tan\theta}{\mu} ~ [{\rm pc}]
\end{align*}

{\noindent}\textbf{Distance (parallax)}:
\begin{align*}
    d = \left(\frac{p}{1''}\right)^{-1} ~ {[\rm pc]}
\end{align*}

{\noindent}\textbf{Distance (photometric)}:
\begin{align*}
    m-M = 5\log\left(\frac{d}{1\,{\rm pc}}\right)-5+A ~ [{\rm mag}]
\end{align*}

{\noindent}\textbf{Distance (spectroscopic)}:
\begin{align*}
    m_V-A_V-M_V = 5\log\left(\frac{d}{1\,{\rm pc}}\right)-5 ~ [{\rm mag}]
\end{align*}

{\noindent}\textbf{Distance modulus}:
\begin{align*}
    m - M \approx 43.17 - 5\log_{10} \left(\frac{H_0}{70\,{\rm km\,s^{-1}\,Mpc}}\right) + 5\log_{10}z + 1.086(1-q_0)z ~ [{\rm mag}]
\end{align*}

{\noindent}\textbf{Distance relation (low redshift)}:
\begin{align*}
    d_p(t_0) \approx d_L \approx \frac{c}{H_0}z ~ [{\rm Mpc}]
\end{align*}

{\noindent}\textbf{Distance-redshift relation (flat Universe)}:
\begin{align*}
    d_L(\kappa=0) = r(1+z) = d_p(t_0)(1+z) ~ [{\rm Mpc}]
\end{align*}

{\noindent}\textbf{Doppler law}:
\begin{align*}
    \frac{v}{c} &= \frac{\Delta\lambda}{\lambda} =  \frac{\lambda_\mathrm{obs}-\lambda_\mathrm{em}}{\lambda_\mathrm{obs}} ~ [{\rm dimensionless}] \\
    \frac{v}{c} &= \frac{-\Delta\nu}{\nu} =  \frac{\nu_\mathrm{em}-\nu_\mathrm{obs}}{\nu_\mathrm{em}} ~ [{\rm dimensionless}]
\end{align*}

{\noindent}\textbf{Einstein relations}:
\begin{align*}
    g_1B_{12} &= g_2B_{21} ~ [{\rm s^{-1}}] \\
    A_{21}    &= \frac{2h\nu^3}{c^2}B_{21}  ~ [{\rm s^{-1}}]
\end{align*}

{\noindent}\textbf{Electron radius (classical)}:
\begin{align*}
    r_{e,0} = \frac{e^2}{m_ec^2} ~ [{\rm cm}]
\end{align*}

{\noindent}\textbf{Elliptical $D_n-\sigma$ relation}:
\begin{align*}
    I_n\left(\frac{D_n}{2}\right)^2\pi &= 2\pi I_eR_e^2 \int\limits_0^{D_n/(2R_e)} ~ [{\rm erg\,s^{-1}\,m^{-2}}] \\
    D_n &= 2.05\times\left(\frac{\sigma_v}{100\,km\,s^{-1}}\right) ~ [{\rm kpc}]
\end{align*}

{\noindent}\textbf{Energy density}:
\begin{align*}
    \Omega \equiv \dfrac{\rho}{\rho_c} = nE ~ [{\rm dimensionless}]
\end{align*}

{\noindent}\textbf{Energy-redshift relation}:
\begin{align*}
    E = E_0(1+z) ~ [{\rm eV}]
\end{align*}

{\noindent}\textbf{Escape velocity}:
\begin{align*}
    v_\mathrm{esc} = \sqrt{\frac{2GM}{r}} ~ [{\rm km\,s^{-1}}]
\end{align*}

{\noindent}\textbf{Extinction coefficient}:
\begin{align*}
    A_\nu &\equiv m-m_0 = -2.5\log(I_\nu/I_{\nu,0}) ~ [{\rm mag}] \\
          &= 2.5\log(e)\tau_\nu ~ [{\rm mag}] \\
          &= 1.086\tau_\nu ~ [{\rm mag}]
\end{align*}

{\noindent}\textbf{Extinction coefficient for the MWG}:
\begin{align*}
    A_{V,\mathrm{MWG}} = (3.1\pm0.1)E(B-V) ~ [{\rm mag}]
\end{align*}

{\noindent}\textbf{Faber-Jackson relation}:
\begin{align*}
    L_\mathrm{FJ} \propto \sigma_v^4 ~ [{\rm erg\,s^{-1}}]
\end{align*}

{\noindent}\textbf{Flux}:
\begin{align*}
    F = \frac{L}{4\pi S_\kappa(r)^2}(1+z)^{-2} ~ [{\rm erg\,s^{-1}\,cm^{-2}}]
\end{align*}

{\noindent}\textbf{Flux (spectral)}:
\begin{align*}
    F_\lambda(t) = \int\limits_0^t \mathrm{SFR}(t-t')S_{\lambda,Z(t-t')}(t')\,\mathrm{d}t' ~ [{\rm erg\,s^{-1}\,cm^{-2}\,Hz^{-1}}]
\end{align*}

{\noindent}\textbf{Friedmann equation}:
\begin{align*}
    \left(\frac{\dot{a(t)}}{a(t)}\right)^2 &=  \left(\frac{H}{H_0}\right)^2 = \frac{8\pi G}{3}\rho - \frac{\kappa c^2}{R_0^2a(t)^2} + \frac{\Lambda}{3} \\
    \left(\frac{H}{H_0}\right)^2 &= \frac{\Omega_{r,0}}{a^4} + \frac{\Omega_{m,0}}{a^3} + \Omega_{\Lambda,0} + \frac{1-\Omega_0}{a^2}
\end{align*}

{\noindent}\textbf{Gravitational potential energy}:
\begin{align*}
    E = -\frac{GMm}{R} ~ [{\rm J}]
\end{align*}

{\noindent}\textbf{Growth factor}:
\begin{align*}
    D_+(t) \propto \frac{H(a)}{H_0}\int\limits_0^a \frac{da'}{[\Omega_m/a'+\Omega_\Lambda a'^2-(\Omega_m+\Omega_\Lambda-1)]^{3/2}} ~ [{\rm dimensionless}]
\end{align*}

{\noindent}\textbf{Gunn-Peterson optical depth}:
\begin{align*}
    \tau_\mathrm{GP} &= \frac{\pi e^2}{m_ec^2}f_\alpha\lambda_\alpha H^{-1}(z)n_\mathrm{HI} ~ [{\rm dimensionless}] \\
    \tau_\mathrm{GP}(z) &= 4.9\times10^5 \left(\frac{\Omega_mh^2}{0.13}\right)^{-1/2} \left(\frac{\Omega_bh^2}{0.02}\right) \left(\frac{1+z}{7}\right)^{3/2} \left(\frac{n_\mathrm{HI}}{n_\mathrm{H}}\right) ~ [{\rm dimensionless}]
\end{align*}

{\noindent}\textbf{Harrison-Zeldovich power spectrum}:
\begin{align*}
    P(k) \propto k^{n_s-1} ~ [{\rm \mu K^2}]
\end{align*}

{\noindent}\textbf{Helium mass fraction}:
\begin{align*}
    Y = \frac{4\times n_n/2}{n_n + n_p}  = \frac{2}{1 + n_n/n_n} ~ [{\rm dimensionless}]
\end{align*}

{\noindent}\textbf{Horizon angular size at recombination}:
\begin{align*}
    \theta_{H,\mathrm{rec}} \approx \sqrt{\frac{\Omega_m}{z_\mathrm{rec}}} \sim \frac{\sqrt{\Omega_m}}{30} \sim \sqrt{\Omega_m}2 ~ [{\rm ^\circ}]
\end{align*}

{\noindent}\textbf{Hubble constant}:
\begin{align*}
    H_0(t) \equiv \dfrac{\dot{a_0}(t)}{a_0(t)} = \dot{a_0}(t) ~ [{\rm km\,s^{-1}\,Mpc}]
\end{align*}

{\noindent}\textbf{Hubble distance}:
\begin{align*}
    d_H \equiv \frac{c}{H(t)} ~ [{\rm Mpc}]
\end{align*}

{\noindent}\textbf{Hubble time}:
\begin{align*}
    t_H \equiv ct_H = \frac{1}{H_0(t)} ~ [{\rm Gyr}]
\end{align*}

{\noindent}\textbf{Hubble's law}:
\begin{align*}
    v = H_0d ~ [{\rm km\,s^{-1}}]
\end{align*}

{\noindent}\textbf{IMF (Salpeter)}:
\begin{align*}
    \xi(m) \propto m^{-2.35} ~ [{\rm units???}]
\end{align*}

{\noindent}\textbf{Jeans mass}:
\begin{align*}
    M_J \equiv \frac{\pi^{5/2}}{6}\left(\frac{c_s^2}{G}\right)^{3/2}\frac{1}{\sqrt{\bar{\rho}}} ~ [{\rm M_\odot}]
\end{align*}

{\noindent}\textbf{Kepler's third law}:
\begin{align*}
    P = \sqrt{\frac{4\pi^2}{G(m_1+m_2)}a^3} ~ [{\rm yr}]
\end{align*}

{\noindent}\textbf{Legendre Polynomials}:
\begin{align*}
    P_0(x) = 1 ~ [{\rm dimensionless}] \\
    P_1(x) = x ~ [{\rm dimensionless}] \\
    P_2(x) = \frac{1}{2}(3x^2-1) ~ [{\rm dimensionless}]
\end{align*}

{\noindent}\textbf{Luminosity (Faber-Jackson)}:
\begin{align*}
    L_\mathrm{FJ} \propto \sigma_v^\alpha ~ [{\rm erg\,s^{-1}}]
\end{align*}

{\noindent}\textbf{Luminosity (Tully-Fisher)}:
\begin{align*}
    L_\mathrm{TF} &= \propto v_\mathrm{max}^\alpha ~ [{\rm erg\,s^{-1}}] \\
    &= \left(\frac{M}{L}\right)^{-2} \left(\frac{1}{G^2\langle I\rangle}\right)v_\mathrm{max}^4 ~ [{\rm erg\,s^{-1}}]
\end{align*}

{\noindent}\textbf{Luminosity distance}:
\begin{align*}
    d_L &\equiv \sqrt{\frac{L}{4\pi F}} ~ [{\rm Mpc}] \\
    d_L(z\ll1) &\approx \frac{c}{H_0}z\left(1+\frac{q_0}{2}z\right) ~ [{\rm Mpc}] \\
    d_L(\kappa=0) &\approx \frac{c}{H_0}z \left(1+\frac{1-q_0}{2}z\right) ~ [{\rm Mpc}]
\end{align*}

{\noindent}\textbf{Luminosity distance-redshift relation}:
\begin{align*}
    d_L = d(1+z) ~ [{\rm Mpc}]
\end{align*}

{\noindent}\textbf{Magnetic monopole density}:
\begin{align*}
    n_M(t_\mathrm{GUT}) \sim \frac{1}{(2ct_\mathrm{GUT})^3} \sim 10^{82} ~ [{\rm m^{-3}}]
\end{align*}

{\noindent}\textbf{Magnitude (absolute)}:
\begin{align*}
    M \equiv -2.5\log_{10}\left(\frac{L}{L_0}\right) ~ [{\rm mag}]
\end{align*}

{\noindent}\textbf{Magnitude (apparent)}:
\begin{align*}
    m \equiv -2.5\log_{10}\left(\frac{F}{F_0}\right) ~ [{\rm mag}]
\end{align*}

{\noindent}\textbf{Matter density-scale factor relation}:
\begin{align*}
    \rho_m = \rho_{m,0}a^{-3} ~ [{\rm g\,cm^{-3}}]
\end{align*}

{\noindent}\textbf{Maxwell-Boltzmann equation}:
\begin{align*}
    n &= g\left(\frac{mk_BT}{2\pi\hbar^2}\right)^{3/2}\exp\left(-\frac{E}{k_BT}\right) ~ [{\rm m^{-3}}] \\
    &= g\left(\frac{mk_BT}{2\pi\hbar^2}\right)^{3/2}\exp\left(-\frac{h\nu}{k_BT}\right) ~ [{\rm m^{-3}}]
\end{align*}

{\noindent}\textbf{Mean free path}:
\begin{align*}
    \ell_\mathrm{mfp} = \dfrac{1}{n\sigma} ~ [{\rm pc}]
\end{align*}

{\noindent}\textbf{Mean molecular weight}:
\begin{align*}
    \mu \equiv \frac{\rho}{nm} ~ [{\rm dimensionless}]
\end{align*}

{\noindent}\textbf{Metallicity index}:
\begin{align*}
    [\mathrm{X}/\mathrm{H}] \equiv \log\left(\frac{n(\mathrm{X})}{n(\mathrm{H})}\right)_* -\log\left(\frac{n(\mathrm{X})}{n(\mathrm{H})}\right)_\odot ~ [{\rm dimensionless}]
\end{align*}

{\noindent}\textbf{Neutrino velocity}:
\begin{align*}
    v_\nu \sim 150(1+z)\left(\frac{m_\nu}{1\,{\rm eV}}\right)^{-1} ~ [{\rm km\,s^{-1}}]
\end{align*}

{\noindent}\textbf{Number density of Lyman-$\alpha$ forest lines}:
\begin{align*}
    N_\alpha(z) \propto (1+z)
\end{align*}

{\noindent}\textbf{Optical depth}:
\begin{align*}
    \tau = \int\limits_0^s \alpha_\nu\mathrm{d}s ~ [{\rm dimensionless}]
\end{align*}

{\noindent}\textbf{Peculiar velocity}:
\begin{align*}
    v_{\rm pec} = c\left(\frac{z-z_\mathrm{cos}}{1+z}\right) ~ [{\rm km\,s^{-1}}]
\end{align*}

{\noindent}\textbf{Photon energy}:
\begin{align*}
    E_\gamma = h\nu ~ [{\rm eV}]
\end{align*}

{\noindent}\textbf{Photon scattering rate (ionized)}:
\begin{align*}
    \Gamma = \dfrac{c}{\ell_\mathrm{mfp}} = n_e\sigma_ec = n_b\sigma_ec = \left(\dfrac{n_{b,0}}{a^3}\right)\sigma_ec = \dfrac{4.4\times10^{-21}{a^3}} ~ [{\rm s^{-1}}]
\end{align*}

{\noindent}\textbf{Planck function}:
\begin{align*}
    B_\nu(T) = \frac{2h\nu^3 }{c^2} \frac{1}{\exp(h\nu/k_BT) - 1} ~ [{\rm erg\,s^{-1}\,cm^{-2}\,Hz^{-1}\,sr^{-1}}]
\end{align*}

{\noindent}\textbf{Polarized angle}:
\begin{align*}
    \chi = \frac{1}{2}\arctan\left(\frac{U}{Q}\right) ~ [{\rm rad}]
\end{align*}

{\noindent}\textbf{Polarized intensity}:
\begin{align*}
    P = \sqrt{Q^2 + U^2} = pI ~ [{\rm Jy}]
\end{align*}

{\noindent}\textbf{Proper distance}:
\begin{align*}
    d_p(t_0) &= c\int\limits_t^{t_0}\frac{\mathrm{d}t}{a(t)} ~ [{\rm Mpc}] \\
    d_p(t_0,z\ll1) &\approx \frac{c}{H_0}z \left(1-\frac{1+q_0}{2}z\right) ~ [{\rm Mpc}]
\end{align*}

{\noindent}\textbf{Proper motion}:
\begin{align*}
    \mu = \frac{v_t}{d} ~ [{\rm arcsec\,yr}]
\end{align*}

{\noindent}\textbf{Proper surface area}:
\begin{align*}
    A_p(t_0) = 4\pi S_\kappa(r)^2 ~ [{\rm Mpc^2}]
\end{align*}

{\noindent}\textbf{Radiation density-scale factor relation}:
\begin{align*}
    \rho_r = \rho_{r,0}a^{-4} ~ [{\rm g\,cm^{-3}}]
\end{align*}

{\noindent}\textbf{Radiative transfer equation}:
\begin{align*}
    \frac{\mathrm{d}I_\nu}{\mathrm{d}s} = j_\nu - \alpha_\nu I_\nu ~ [{\rm erg\,s^{-1}\,cm^{-2}\,Hz^{-1}\,sr^{-1}\,pc^{-1}}]
\end{align*}

{\noindent}\textbf{Random walk}:
\begin{align*}
    d = \sqrt{N}\ell ~ [{\rm m}]
\end{align*}

{\noindent}\textbf{Rayleigh-Jeans limit}:
\begin{align*}
    I_\nu = \frac{2k_BT\nu^2}{c^2} ~ [{\rm erg\,s^{-1}\,cm^{-2}\,Hz^{-1}\,sr^{-1}}]
\end{align*}

{\noindent}\textbf{Redshift of thermal coupling}:
\begin{align*}
    z_t \approx 140\left(\frac{\Omega_bh^2}{0.022}\right)^{2/5} ~ [{\rm dimensionless}]
\end{align*}

{\noindent}\textbf{Redshift-wavelength-scalefactor relation}:
\begin{align*}
    \frac{1}{1+z} = \frac{\lambda}{\lambda_0} = a ~ [{\rm dimensionless}]
\end{align*}

{\noindent}\textbf{Relaxation time}:
\begin{align*}
    t_\mathrm{relax} &\approx \frac{R}{6v} \frac{N}{\ln(N/2)} ~ [{\rm yr}] \\
    t_\mathrm{relax} &= \frac{t_\mathrm{cross}}{6} \frac{N}{\ln(N/2)} ~ [{\rm yr}]
\end{align*}

{\noindent}\textbf{Robertson-Walker metric}:
\begin{align*}
    \mathrm{d}s^2 = c\mathrm{d}t^2 -a(t)^2 \left( \frac{\mathrm{d}x^2}{1-\kappa x^2/R^2} + x^2\Omega^2 \right)
\end{align*}

{\noindent}\textbf{Rotational velocity (at Solar position)}:
\begin{align*}
    V_0 = \sqrt{\frac{GM(<R)}{R_0}} ~ [{\rm m\,s^{-1}}]
\end{align*}

{\noindent}\textbf{Saha equation}:
\begin{align*}
    \frac{n_{i+1}}{n_i} = T^{3/2}\mathrm{e}^{(-1/T)} ~ [{\rm dimensionless}]
\end{align*}

{\noindent}\textbf{Scale factor evolution (radiation-dominated)}:
\begin{align*}
    a_m(t) \propto t^{1/2} ~ [{\rm dimensionless}]
\end{align*}

{\noindent}\textbf{Scale factor evolution (matter-dominated)}:
\begin{align*}
    a_m(t) \propto t^{2/3} ~ [{\rm dimensionless}]
\end{align*}

{\noindent}\textbf{Scale factor evolution ($\Lambda$-dominated)}:
\begin{align*}
    a_\Lambda(t) \propto e^{Ht} ~ [{\rm dimensionless}]
\end{align*}

{\noindent}\textbf{Scale factor-redshift relation}:
\begin{align*}
    a = \dfrac{1}{1+z} ~ [{\rm dimensionless}]
\end{align*}

{\noindent}\textbf{Schmidt-Kennicutt law}:
\begin{align*}
    \frac{\Sigma_\mathrm{SFR}}{{\rm M_\odot\,year^{-1}\,kpc^{-2}}} = (2.5\pm0.7)\times10^{-4} \left(\frac{\Sigma_\mathrm{gas}}{{\rm M_\odot\,pc^{-2}}}\right)^{1.4\pm0.15}
\end{align*}

{\noindent}\textbf{Schwarzschild radius}:
\begin{align*}
    r_S \equiv \frac{2GM}{c^2} = 2.95\left(\frac{M}{{\rm M_\odot}}\right) ~ [{\rm km}]
\end{align*}

{\noindent}\textbf{Solid angle}:
\begin{align*}
    \Omega = \frac{A}{r^2} ~ [{\rm sr}]
\end{align*}

{\noindent}\textbf{Sound speed}:
\begin{align*}
    c_s &= \sqrt{\frac{\partial P}{\partial\rho}} ~ [{\rm m\,s^{-1}}] \\
    c_s &\sim \sqrt{\frac{k_BT}{\mu m}} ~ [{\rm m\,s^{-1}}]
\end{align*}

{\noindent}\textbf{Spin temperature}:
\begin{align*}
    T_s^{-1} = \frac{T_\gamma^{-1} + x_\alpha T_\alpha^{-1} + x_cT_K^{-1}}{1+x_\alpha+x_c} ~ [{\rm K^{-1}}]
\end{align*}

{\noindent}\textbf{Spin temperature of $21\,{\rm cm}$}:
\begin{align*}
    \frac{n_1}{n_2} = \frac{g_1}{g_2}\exp\left(-\frac{0.068\,{\rm K}}{T_s}\right) ~ [{\rm dimensionless}],
\end{align*}

{\noindent}\textbf{Star formation history}:
\begin{align*}
    \psi(z) = 0.015\frac{(1+z)^{2.7}}{1+[1+z/2.9]^{5.6}} ~ [{\rm M_\odot\,year^{-1}\,Mpc^{-3}}].
\end{align*}

{\noindent}\textbf{Star formation rate}:
\begin{align*}
    \mathrm{SFR} = -\frac{\mathrm{d}M_\mathrm{gas}}{\mathrm{d}t} ~ [{\rm M_\odot\,yr^{-1}}]
\end{align*}

{\noindent}\textbf{Star formation rate per unit area}:
\begin{align*}
    \Sigma_\mathrm{SFR} \propto \Sigma_\mathrm{gas}^{1.4} ~ [{\rm M_\odot\,year^{-1}\,pc^{-2}}]
\end{align*}

{\noindent}\textbf{Stefan-Boltzmann constant}:
\begin{align*}
    \sigma = \frac{2\pi^5k^4}{15c^2h^3} ~ [{\rm W\,m^{-2}\,K^{-4}}]
\end{align*}

{\noindent}\textbf{Stefan-Boltzmann Law}:
\begin{align*}
    F = \sigma T^4 ~ [{\rm W\,m^{-2}}]
\end{align*}

{\noindent}\textbf{Sunyaev-Zeldovich temperature fluctuation}:
\begin{align*}
    \frac{\Delta T_\mathrm{SZE}}{T_\mathrm{CMB}} = f(x)y = f(x)\int n_e\frac{k_BT_e}{m_ec^2}\sigma_T \mathrm{d}l ~ [{\rm dimensionless}] \\
    \frac{\Delta T_\mathrm{SZ}}{T_\mathrm{CMB}} = -\tau_e \left(\frac{v_\mathrm{pec}}{c}\right) ~ [{\rm dimensionless}]
\end{align*}

{\noindent}\textbf{Surface brightness (bulge)}:
\begin{align*}
    \mu_\mathrm{bulge} = \mu_e + 8.3268\left[\left(\frac{R}{R_e}\right)^{1/4}-1\right] ~ {\rm mag\,arcsec^{-2}}
\end{align*}

{\noindent}\textbf{Surface brightness (disk)}:
\begin{align*}
    \mu_\mathrm{disk} = \mu_0 + 1.09\left(\frac{R}{h_R}\right) ~ {\rm mag\,arcsec^{-2}}.
\end{align*}

{\noindent}\textbf{Temperature-redshift-scalefactor relation}:
\begin{align*}
    T = (1+z)T_0 = \frac{T_0}{a} ~ [{\rm K}]
\end{align*}

{\noindent}\textbf{Temperature-scale factor relation}:
\begin{align*}
    T = aT_0 ~ [{\rm K}]
\end{align*}

{\noindent}\textbf{Thomson cross section}:
\begin{align*}
\sigma_T = \frac{8\pi}{3}r_{e,0}^2 ~ [{\rm m^2}]
\end{align*}

{\noindent}\textbf{Tired Light Hypothesis energy scaling}:
\begin{align*}
    E = E_0 \mathrm{e}^{(-r/R_0)} ~ [{\rm eV}]
\end{align*}

{\noindent}\textbf{Tully-Fisher relation}:
\begin{align*}
    L_\mathrm{TF} \propto v_\mathrm{max}^4 ~ [{\rm erg\,s^{-1}}]
\end{align*}

{\noindent}\textbf{Uncertainty Principle}:
\begin{align*}
    \sigma_x\sigma_p \geq \frac{\hbar}{2} ~ [{\rm J\,s}] \\
    \Delta E \Delta t \geq \frac{\hbar}{2} ~ [{\rm J\,s}]
\end{align*}

{\noindent}\textbf{Velocity (circular)}:
\begin{align*}
    v_c = \sqrt{\frac{GM}{r}} ~ [{\rm m\,s^{-1}}]
\end{align*}

{\noindent}\textbf{Velocity (radial)}:
\begin{align*}
    v_r = \left(\frac{\Delta\lambda}{\lambda_0}\right)c ~ [{\rm m\,s^{-1}}]
\end{align*}

{\noindent}\textbf{Velocity (tangential)}:
\begin{align*}
    v_t = d\mu = 4.74\left(\frac{d}{1\,{\rm pc}}\right) \left(\frac{\mu}{1\,''\,{\rm year^{-1}}}\right) ~ [{\rm m\,s^{-1}}]
\end{align*}

{\noindent}\textbf{Virial theorem}:
\begin{align*}
    E_\mathrm{pot} &= -2E_\mathrm{kin} \\
    E_\mathrm{tot} &= \frac{1}{2}E_\mathrm{pot}
\end{align*}

{\noindent}\textbf{Wavelength-redshift relation}:
\begin{align*}
    \lambda_0 = \frac{1}{a(t)}\lambda = (1+z)\lambda ~ [{\rm m}]
\end{align*}

{\noindent}\textbf{Wien's Law}:
\begin{align*}
    B_\nu(T) &= \frac{2h\nu^3}{c^2} \frac{1}{\exp(h\nu/k_BT) - 1} ~ [{\rm erg\,s^{-1}\,cm^{-2}\,Hz^{-1}\,sr^{-1}}] \\
    B_\nu(T) &\approx \frac{2h\nu^3}{c^2} \frac{1}{\exp(h\nu/k_BT)} ~ [{\rm erg\,s^{-1}\,cm^{-2}\,Hz^{-1}\,sr^{-1}}] \\
    B_\nu(T) &\approx \frac{2h\nu^3}{c^2} \exp\left(\frac{-h\nu}{k_BT}\right) ~ [{\rm erg\,s^{-1}\,cm^{-2}\,Hz^{-1}\,sr^{-1}}]
\end{align*}






































%%%%%%%%%%%%%%%%%%%%%%%%%%%%%%%%%%%%%%%%%%%%%%%%%%%%%%%%
%                                                      %
%                                                      %
%                   GENERAL NOTES                      %
%                                                      %
%                                                      %
%%%%%%%%%%%%%%%%%%%%%%%%%%%%%%%%%%%%%%%%%%%%%%%%%%%%%%%%

\newpage
\subsection{General Notes}

\begin{itemize}
    \item $k_B(300\,\mathrm{K}) \sim \dfrac{1}{40}\,{\rm eV}$
    \item $k_B(11,000\,\mathrm{K}) \sim 1\,{\rm eV}$
    \item $k_B(10^7\,\mathrm{K}) \sim 1\,{\rm keV}$
    \item Systems in thermal equilibrium satisfy normal populations of $(n_1/g_1)>(n_2/g_2)$; inverted populations satisfy $(n_1/g_1)<(n_2/g_2)$ (e.g., masers).
    \item If a particle scatters with a rate greater than the expansion rate of the Universe, that particle remains in equilibrium.
    \item Ordinary (i.e., baryonic) matter contributes at most 5\% of the Universe's critical density.
    \item Multipole moment $\ell=1$ and $\ell=2$ are the dipole and quadrupole moment, respectively.
    \item The greatest baryon contribution to the density comes not from stars in galaxies, but rather from gas in groups of galaxies; in these groups, $\Omega\sim0.02$.
    \item The ratio of the neutrino temperature $T_\nu$ to the CMB temperature $T_\mathrm{CMB}$ is $\dfrac{T_\nu}{T_\mathrm{CMB}} = \left( \dfrac{4}{11} \right)^{1/3}$.
    \item We expect photons to decouple from matter when the Universe is already well into the matter-dominated era (i.e., not in the radiation-dominated era).
    \item When the Universe was only one second old, the mean-free-path of a photon was about the size of an atom.
    \item Nuclear binding energies are typically in the $\rm{MeV}$ range, which explains why Big Bang nucleosynthesis occurs at temperatures a bit less than $1\,\rm{MeV}$ even though nuclear masses are in the $\rm{GeV}$ range.
    \item The electron-positron energy threshold is $\sim10^{10}~{\rm K}$, above which there exist thermal populations of both electrons and neutrinos to make the reaction $p+e^-\leftrightarrow n+\nu$ go equally well in either direction.
    \item Themodynamically, nucleons with greater binding energies are more energetically favourable.
    \item Simplest way to form helium is via deuterium fusion rather than the improbable coincidence of $2$ protons and $2$ neutrons all arriving at the same place simultaneously to make $^4$He in one go.
    \item Nucleosynthesis starts at about $10^{10}\,{\rm K}$ when the Universe was about $1\,{\rm s}$ old, and effectively ends when it has cooled by a factor of $10$, and is about $100$ times older.
    \item Primordial elemental abundances: $75\%$ H, $25\%$ He, trace Li.
    \item A normal population of states in LTE satisfies $(n_1/g_1)/(n_2/g_2)>1$ via the Maxwell-Boltzmann equation; inverted populations such as masers satisfy $(n_1/g_1)/(n_2/g_2)<1$.
    \item The solid angle subtended by a complete sphere is $4\pi~{\rm sr}$.
    \item Just as the period of a Cepheid tells you its luminosity, the rise and fall time of a type Ia SN tells you its peak luminosity.
    \item The average Type 1a has a peak luminosity of $L=4\times10^9\,\mathrm{L}_\odot$ which is 100,000 times brighter than even the brightest Cepheid variable.
    \item The Hubble distance $d_H$ is the distance between the Earth and any astrophysical object receding away from us at the speed of light.
    \item At $z=10^9$, the Jeans mass is $M_J\sim M_\odot$; at $z=10^6$, the Jeans mass is $M_J\sim M_\mathrm{gal}$.
    \item Apparent magnitudes $0<m<6$ are typically visible to the naked eye.
    \item The absolute magnitude of a light source $M$ is defined as the apparent magnitude $m$ that it would have if it were at a luminosity distance of $d_L = 10\,{\rm pc}$.
    \item The apparent magnitude is really nothing more than a logarithmic measure of the flux, and the absolute magnitude is a logarithmic measure of the luminosity.
    \item When space is positively curved, the proper surface area is $A_p(t_0) < 4\pi r^2$, and the photons from distance $r$ are spread over a smaller area than they would be in flat space. When space is negatively curved, $A_p(t_0) > 4\pi r^2$, and photons are spread over a larger area than they would be in flat space.
    \item The Benchmark Model has a deceleration parameter of $q_0\approx−0.55$.
    \item The angular distribution of the CMB temperature reflects the matter inhomogeneities at the redshift of decoupling of radiation and matter.
    \item The CMB is linearly polarized at the 10\% level.
    \item CMB polarization probes the epoch of last scattering directly as opposed to the temperature fluctuations which may evolve between last scattering and the present. Moreover, different sources of temperature anisotropies (scalar, vector, and tensor) give different patterns in the polarization: both in its intrinsic structure and in its correlation with the temperature fluctuations themselves.
    \item The sound speed in the photon-dominated fluid of the early Universe is given by $c_s\approx c/\sqrt{3}$. Thus, the sound horizon is about a factor of $\sqrt{3}$ smaller than the event horizon at this time.
    \item At recombination, the free electrons recombined with the hydrogen and helium nuclei, after which there are essentially no more free electrons which couple to the photon field. Hence, after recombination the baryon fluid lacks the pressure support of the photons, and the sound speed drops to zero -- the sound waves do no longer propagate, but get frozen in.
    \item In a universe which is dominated by dark matter the expected CMB fluctuations on small angular scales are considerably smaller than in a purely baryonic universe.
    \item The number density of each of the three flavors of neutrinos ($\nu_e$, $\nu_\mu$, and $\nu_\tau$) has been calculated to be 3/11 times the number density of CMB photons. This means that at any moment, about twenty million cosmic neutrinos are zipping through your body.
    \item Due to the steep slope of the IMF, most of the stellar mass is contained in low-mass stars; however, since the luminosity of main-sequence stars depends strongly on mass, approximately as $L\propto M^3$, most of the luminosity comes from high-mass stars.
    \item Cooling by the primordial gas is efficient only above $T\gtrsim2\times10^4\,{\rm K}$ since metal lines cannot contribute to the cooling.
    \item Whereas in enriched gas molecular hydrogen is formed on dust particles, the primordial gas had no dust so H$_2$ must form in the gas phase itself, rendering its abundance very small.
    \item Even a tiny neutral fraction, $X_\mathrm{HI}\sim10^{-4}$, gives rise to complete GP absorption due to the large absorption cross section of Ly$\alpha$.
    \item The key to the detectability of the $21\,{\rm cm}$ signal hinges on the spin temperature $T_s$. Only if this temperature deviates from the background temperature, will a signal be observable.
    \item The spin temperature becomes strongly coupled to the gas temperature when $x_\mathrm{tot}\equiv x_c+x_\alpha\gtrsim1$ and relaxes to $T_\gamma$ when $x_\mathrm{tot}\ll1$.
    \item The ratio of black hole mass and bulge mass is approximately $1/300$. In other words, 0.3\% of the baryon mass that was used to make the stellar population in the bulge of these galaxies was transformed into a central black hole.
    \end{itemize}

\end{document}
